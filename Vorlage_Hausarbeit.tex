\documentclass[a4paper,12pt,parskip,bibtotoc,liststotoc]{article}
    %Festlegung der Dokumentenklasse, zahlreiche Vereinbarungen über Layout, Gliederungsstrukturen,
    %bsp. article -> section, subsection..., book -> chapter, section...
    %parskip = Abstand zwischen Absätzen, Veränderung durch \setlength
\usepackage[utf8]{inputenc}   %Eingabezeichencodierung, die direkte Tastatureingabe von Umlauten ist möglich
\usepackage[ngerman]{babel}     %Neu deutsche Rechtschreibung, Umlaute können geschrieben werden
\usepackage{setspace}           %für Zeilenabstand
\usepackage[notindex,nottoc]{tocbibind}   %Inhaltsverzeichnisse erstellen

%zusätzliche benötigte Pakete
\usepackage{graphicx}           %Graphik
\usepackage{amsmath,amssymb}    %Mathematik
\usepackage{natbib}             %Zitate
\usepackage{marvosym}           %enthält Symbole wie das Eurozeichen
\usepackage{eurosym}
%\setcounter{secnumdepth}{3}
%\setcounter{tocdepth}{3}
\usepackage{footmisc}
\usepackage{listings}
\usepackage{color}
\usepackage{textcomp}
\definecolor{listinggray}{gray}{0.9}
\definecolor{lbcolor}{rgb}{0.9,0.9,0.9}
\lstdefinestyle{befehl}{
	backgroundcolor=\color{lbcolor},
	tabsize=4,
	rulecolor=,
	language=ruby,
        upquote=true,
        aboveskip={0.5\baselineskip},
        columns=fixed,
        showstringspaces=false,
        extendedchars=true,
        breaklines=true,
        prebreak = \raisebox{0ex}[0ex][0ex]{\ensuremath{\hookleftarrow}},
        showtabs=false,
        showspaces=false,
        showstringspaces=false,
        identifierstyle=\ttfamily,
        keywordstyle=\color[rgb]{0,0,1},
        commentstyle=\color[rgb]{0.133,0.545,0.133},
        stringstyle=\color[rgb]{0.627,0.126,0.941}
}

\lstdefinestyle{Listing}{
	frame=single,
	tabsize=4,
        extendedchars=true,
        basicstyle=\footnotesize,
	rulecolor=,
	language=ruby,
        upquote=true,
        aboveskip={0.5\baselineskip},
        columns=fixed,
        showstringspaces=false,
        extendedchars=true,
        breaklines=true,
        prebreak = \raisebox{0ex}[0ex][0ex]{\ensuremath{\hookleftarrow}},
        showtabs=false,
        showspaces=false,
        showstringspaces=false,
        identifierstyle=\ttfamily,
        keywordstyle=\color[rgb]{0,0,1},
        commentstyle=\color[rgb]{0.133,0.545,0.133},
        stringstyle=\color[rgb]{0.627,0.126,0.941}
}

\lstdefinestyle{Listing2}{
	frame=single,
	tabsize=4,
        extendedchars=true,
        basicstyle=\footnotesize,
	rulecolor=,
	language=Java,
        upquote=true,
        aboveskip={0.5\baselineskip},
        columns=fixed,
        showstringspaces=false,
        extendedchars=true,
        breaklines=true,
        prebreak = \raisebox{0ex}[0ex][0ex]{\ensuremath{\hookleftarrow}},
        showtabs=false,
        showspaces=false,
        showstringspaces=false,
        identifierstyle=\ttfamily,
        keywordstyle=\color[rgb]{0,0,1},
        commentstyle=\color[rgb]{0.133,0.545,0.133},
        stringstyle=\color[rgb]{0.627,0.126,0.941}
}

\lstset{literate=%
    {Ö}{{\"O}}1
    {Ä}{{\"A}}1
    {Ü}{{\"U}}1
    {ß}{{\ss}}1
    {ü}{{\"u}}1
    {ä}{{\"a}}1
    {ö}{{\"o}}1
    {~}{{\textasciitilde}}1
}
\usepackage{amsmath}
%\usepackage{mathtools} % lädt auch amsmath
\usepackage{moreverb}

\usepackage{amsthm}
\newtheorem{mydef}{Definition}

\usepackage[table]{xcolor}
\usepackage{colortbl}
\usepackage{multirow}
\usepackage{mdwlist}   %Verringerung Abstand zwischen items -> \begin{itemize*} \end{itemize*}
\usepackage[labelsep=space,justification=centering]{caption}

%\usepackage{hyperref}  %erlaubt Links innerhalb des pdf-Dokuments zu erzeugen

\setlength{\parindent}{0pt}     %Verhinderung des horizontalen Einrückens zu Beginn eines Absatzes

%Seitenlayout
\topmargin -0.9cm       %Vertikaler Abstand der Kopfzeile von der Bezugslinie
\textheight 25cm        %Abstand der Grundlinie der Kopfzeile zum Haupttext
\textwidth 16.5cm       %Breite des Haupttexts
\footskip 1cm           %Abstand der Grundlinien der letzten Textzeile und der Fußzeile
\voffset -0.5cm         %Vertikale Bezugspunktposition
\hoffset -1.2cm         %Horizontale Bezugspunktposition

\onehalfspacing         %anderthalbzeiliger Abstand

\newcommand{\url}{\;}   %URL im Literaturverzeichnis

%eigene Befehlsdefinitionen
\newcommand{\be}{\begin{equation}}     %Mathematische Umgebung
\newcommand{\ee}{\end{equation}}
\newcommand{\bea}{\begin{eqnarray}}
\newcommand{\eea}{\end{eqnarray}}
\newcommand{\bean}{\begin{eqnarray*}}  %ohne Nummerierung
\newcommand{\eean}{\end{eqnarray*}}    %ohne Nummerierung
%%%%%%%%%%%%%%%%%%%%%%%%%%%%%%%%%%%%%%%%%%%%%%%%%%


%%%%%%%%%%%%%%%%%%%%%%%%%%%%%%%%%%%%%%%%%%%%%%%%%%%%%%
%
%    Anfang des Textes
%
%%%%%%%%%%%%%%%%%%%%%%%%%%%%%%%%%%%%%%%%%%%%%%%%%%%%%%
\begin{document}

\pagenumbering{roman}  %römische Seitennummerierung
%%%%%%%%%%%%%%%%%%%%%%%%%%%%%%%%%%%%%%%%%%%%%%%%%%%%%%
%
%    Titelseite
%
%%%%%%%%%%%%%%%%%%%%%%%%%%%%%%%%%%%%%%%%%%%%%%%%%%%%%%
\thispagestyle{empty}  %keine Seitenzahl auf Titelseite
Leibniz Universität Hannover\\
Wirtschaftswissenschaftliche Fakultät\\
Institut für Produktionswirtschaft\\
Prof.\ Dr.\ Stefan Helber

\vspace{5cm}

\begin{center}
Hausarbeit im Rahmen der Veranstaltung \\
Entwicklung von Anwendungssystemen  im WiSe 2014/2015 \\
(Veranstaltungs-Nr. 173610)

\vspace{2.5cm}

%Thema Nr. 5\\[1mm]    %hier Themennummer eintragen
{\Large RCPSP \\
RCPSP}
\end{center}

\vspace{5.5cm}


\begin{table}[h!]
    \vspace*{-3mm}
    \hspace*{2mm}
  \renewcommand{\arraystretch}{1,5}
    \begin{tabular}{ll}
Andreas Hipp &Robert Matern \\
Ungerstr. 24&Plathnerstr. 49 \\
30451 Hannover&30175 Hannover \\
Matr.-Nr. 3027520 ???&Matr.-Nr. 2798160 \\[3mm]
Abgabedatum: 24.03.2015
	\end{tabular}
\end{table}

\newpage

%Inhaltsverzeichnis erstellen
\tableofcontents

\newpage  %neue Seite

%Abbildungsverzeichnis erstellen
\listoffigures

%Tabellenverzeichnis erstellen
\listoftables

%Codeverzeichnis
\renewcommand\lstlistlistingname{Quellcodeverzeichnis} 
\lstlistoflistings 
\renewcommand*\lstlistingname{Quellcode} 

\newpage
%Abkürzungsverzeichnis
\section*{Abkürzungsverzeichnis}
\addcontentsline{toc}{section}{Abkürzungsverzeichnis}
\begin{table}[h!]
    \vspace*{-3mm}
    \hspace*{2mm}
  \renewcommand{\arraystretch}{1,5}
    \begin{tabular}{ll}  %hier die Spaltenausrichtung und Anzahl eintragen
           RCPSP      & Resource-Constrained Project Scheduling \\
           RoR & Ruby on Rails \\
SGS & Schedule Generation Scheme\\
	\end{tabular}
\end{table}
\newpage
%Symbolverzeichnis
\section*{Symbolverzeichnis}
\addcontentsline{toc}{section}{Symbolverzeichnis}
\begin{table}[h!]
    \vspace*{-3mm}
        \hspace*{2mm}
      \renewcommand{\arraystretch}{1,5}
    \begin{tabular}{ll} 
$d_i$ & Dauer von Vorgang $i$ \\
$FE_i$& frühestes Ende von Vorgang $i$\\
$i,h=1,...,I$ & Vorgänge \\
$k_{ir}$& Kapazitätsbedarf von Vorgang $i$ auf Ressource $r$\\
$kp_r$ & verfügbare Kapazität von Ressource $r$ je Periode\\
$\mathcal{N}_i$ & Menge der direkten Nachfolger von Vorgang $i$ \\
$oc_r$ & Kosten einer Einheit Zusatzkapazität von Ressource $r$ \\
$O_{rt}$ & Zusatzkapazität von Ressource $r$ in Periode $t$ \\
$r=1,...,R$ & Ressourcen \\
$SE_i$& spätestes Ende von Vorgang $i$\\
$t,\tau=1,..., T$ & Perioden\\
$\mathcal{V}_i$ & Menge der direkten Vorgänger von Vorgang $i$ \\
$X_{it}\in\{0,1\}$ & gleich $1$, falls Vorgang $j$ in Periode $t$ endet, sonst $0$
  	\end{tabular}
\end{table}
\newpage
\pagenumbering{arabic}   %ab hier arabische Seitenzahlen beginnend mit 1

%%%%%%%%%%%%%%%%%%%%%%Textteil%%%%%%%%%%%%%%%%%%%%%%%%%%

\section{Einleitung} \label{start}
%Bereits seit mehreren Dekaden spielt Projektarbeit eine wichtige Rolle bei der Aufgabenabwicklung in Wirtschaft und Verwaltung.\footnote{Vgl. \cite{zimmermann2006projektplanung}, S. VI}
Bei einem Projekt handelt es sich um eine zeitlich befristete, relativ innovative und risikobehaftete Aufgabe von erheblicher Komplexität, die meist einer gesonderten Planung bedarf.\footnote{Vgl. \cite{projektdef}} Dementsprechend von großer Bedeutung ist die vorhergehende und genaue Planung von Projekten.\footnote{Vgl. \cite{zimmermann2006projektplanung}, S. VI\label{zum}} Projektplanung ist die Planung aller Arbeitsgänge eines Projekts durch Zuweisung eines Startzeitpunktes, so dass die Zeitbeziehung zwischen den Vorgängen eingehalten und knappe Ressourcenkapazitäten nicht überschritten werden.\footref{zum} Durch das Zerlegen des Projekts in einzelne Arbeitsgänge wird versucht die Komplexität zu reduzieren und eine geordnete Abfolge der Arbeitsgänge zu erstellen, um das Projektziel zu erreichen.\footnote{Vgl. \cite{zimmermann2006projektplanung}, S. 4} Projektziele können dabei unterschiedlich kategorisiert werden, z. B. in Sach-, Termin- oder Kostenziele.\footnote{Vgl. \cite{felkai2011analysieren}, S. 52}\\

Nach DIN 69900 hat ein Arbeitsgang oder ein einzelner Vorgang eines Projekts einen definierten Anfang sowie ein definiertes Ende und dient für das Projekt als Ablaufelement zur Beschreibung eines bestimmten Geschehens.\footnote{Vgl. \cite{69900D}, S. 15} Trotz der Zerlegung besitzen die einzelnen Arbeitsgänge des Projekts eine Beziehung, mit der die Reihenfolge der Ablauffolge bestimmbar ist.\footnote{Vgl. \cite{kellenbrink2014einfuhrung}, S. 6-7} Oft wird zur Darstellung der Vorgangsrelationen ein Vorgangsknoten-Netzplan verwendet.\footnote{?????} %D. h. es können nur Arbeitsgänge abgeschlossen werden, wenn deren notwendigen Vorgänge bereits abgeschlossen wurden. Ein einfaches Beispiel wäre die sogenannte Hochzeit in der Automobilherstellung. Sobald Karosserie und Motor eines Fahrzeugs hergestellt wurden, können diese zwei Elemente in einem nachfolgenden Prozess verbunden werden.
Ein Arbeitsgang ist i. d. R. verbunden mit dem Einsatz von Ressourcen, welche wiederum mit Kosten verbunden sind. Eine Möglichkeit, das Projektziel unter minimaler Ressourcenverwendung zu erreichen, ist die effiziente Planung der Ablauffolge der Arbeitsgänge eines Projekts.\footnote{Vgl. \cite{bartels2009projektplanung}, S. 11-12} Damit ist es möglich, mehrere Projekte bei einer gegebenen Zeitvorgabe unter Einhaltung von Ressourcenrestriktionen fertigzustellen bzw. bei konstanter Ressourcenkapazität ein Projekt in kürzerer Zeit abzuschließen. \\

%Dementsprechend versucht eine effiziente Projektplanung, neben der Einhaltung des Projektziels, auch den Einsatz der Ressourcen zu minimieren.\footnote{Vgl. \cite{bartels2009projektplanung}, S. 11-12}  %Projektmanagement, neben der Einhaltung des Projektziels, auch den Einsatz der Ressourcen zu minimieren.\footnote{Vgl. \cite{bartels2009projektplanung}, S. 11-12}


%\begin{mydef}
%\glqq Allgemein bezeichnet der Begriff Projektmanagement alle leitenden und administrativen Aktivitäten, die zur Durchführung eines Projektes notwendig sind. Es beschreibt die Gesamtheit von Führungsaufgaben, -organi\-sation, -techniken und -mitteln zur zielorientierten Durchführung großer Vorhaben.\grqq\footnote{Vgl. \cite{hering2014projektmanagement}, S. 1-3; in Anlehnung an DIN 69901 und ISO 21500:2012-09}
%\end{mydef}

%Beispielweise kann ein Automobilhersteller durch Optimierung des Produktentstehungsprozess durch effizientes planen seiner vorhandenen Ressourcen in einer vordefinierten Zeit eine größere Anzahl an Fahrzeugen entwickeln, als wenn das Unternehmen keine effiziente Projektplanung verfolgt. 

%Eine effiziente Projektplanung reduziert die gesamte Fertigstellungsdauer eines Projekts. Es wird eine wirkungsvolle Gestaltung des Verbrauchs der zur Verfügung stehenden Ressourcen über die Laufzeit des Projekts ermöglicht. Somit handelt es sich um ein mathematisches Optimierungsproblem, bei dem für ein Projekt die Ressourcenbeschränkung über die Laufzeit einzuhalten ist und die Fertigstellungszeit minimiert werden soll.
%Für das ressourcenbeschränkte Projektplanungsproblem gibt es Verfahren der exakten und heuristischen Lösung. In der Praxis, wie auch in dieser Arbeit \footnote{Vgl. Kapitel \ref{notwendig}}, wird i. d. R. auf Heuristiken zurückgegriffen.\footnote{Vgl. \cite{herroelen2005project}, S. 420}
%Eine typische Ressource in der Projektplanung ist der Faktor Zeit, da lt. Definition ein Projekt zeitlich befristet ist. % Andere Arten von Ressourcen sind dürfen bei der Planung von Projekten jedoch nicht vernachlässigt werden.
Zur Bestimmung der optimalen Ablauffolge der einzelnen Arbeitsgänge eines Projekts kann ein Optimierungsmodell verwendet werden, bei der für eine festgelegten Ablauffolge eines Projekts und unter Berücksichtigung der Ressourcenbeschränkung die Fertigstellungszeit minimiert wird. Im Kapitel \ref{Grund} wird eine solche Modellformulierung für das ressourcenbeschränkte Projektplanungsproblem als sogenannte Kapazitätsplanung vorgestellt.\footnote{????} Alternativ wird in dem Kapitel das Optimierungsmodell um die Bedienung erweitert, dass  Zusatzkapazitätseinheiten gebucht werden können. Mit dieser Modellerweiterung wird von der Kostenplanung in Projekten gesprochen. Bezeichnet wird im Allgemeinen das ressourcen-beschränkte Projektplanungsproblem  mit der englischen Bezeichnung des \textit{Resource-Constrained Project Scheduling Problem (RCPSP)}. Bei dem RCPSP handelt es sich um eine abstrakte mathematische Modellformulierung. Ziel der vorliegenden Arbeit ist es das RCPSP in Ruby on Rails (RoR) zu implementieren. Bei RoR handelt es sich um ein Framework zur Entwicklung von Webdokumenten bzw. Internetseiten.\footnote{???} Es baut auf der Programmiersprache Ruby auf und ist ursprünglich von David Heinemeier Hansson entwickelt.\footnote{???} Die Implementierung bedarf einer Verknüpfung von RoR und GAMS\footnote{General Algebraic Modeling System}. Unter GAMS wird eine algebraische Modellierungssprache für mathematische Optimierungsprobleme verstanden, mit der das RCPSP gelöst wird.\footnote{???} Im Kapitel \ref{Haupt} wird die Entwicklung des Anwendungssystems zum Lösen des RCPSP ausführlich beschrieben. Ergänzt wird diese Arbeit durch eine kritische Würdigung des Anwendungssystems in Kapitel \ref{krit} sowie einem Fazit in Kapitel \ref{Fazit}.

\section{Grundlagen zur ressourcen-beschränkten Projektplanung und zu dem Framework Ruby on Rails} \label{Grund}
\subsection{Kapazitätsplanung}
Ein Großteill an Projekten besitzt die Eigenschaft eines beschränkten Ressourcenkontingents.\footnote{Vgl. \cite{kellenbrink2014einfuhrung}, S. 11} Soll demgemäß die vorgegebene Terminierung des Projektes als zuvor festgesetztes Ziel erreicht werden, muss neben der Reihenfolgerestriktion auch der Ressourcenbedarf der unterschiedlichen Arbeitsgänge sichergestellt werden. Mit der Einhaltung des Ressourcenbedarfs ist es möglich, alle zur Erfüllung des Projektes notwendigen Arbeitsgänge auszuführen und somit letztendlich das Projekt abzuschließen. Neben limitierten Ressourcen, die während des gesamten Projekts nur ein Mal zur Verfügung stehen, wie bspw. das Projektbudget, gibt es Ressourcen, die nach einer bestimmten Anzahl von Perioden erneuert werden können.\footnote{Vgl. \cite{neumann2003project}, S. 21-22} Erneuerbare Ressourcen sind bspw. die Produktionskapazität einer Maschine oder der Personaleinsatz für ein Projekt. In dieser Arbeit wird der Fokus auf diese erneuerbaren Ressourcen gelegt.\\

Zur Lösung des ressourcenbeschränkten Projektplanungsproblems kann das Modell RCPSP genutzt werden. Das RCPSP legt durch Fixierung der Aktivitätsstartzeitpunkte den Projektgrundablauf zur Zielerreichung der Minimierung der Projektdauer fest. Dies geschieht unter Einhaltung der Startzeitpunkt- bzw. der Vorrangsbedingung der einzelnen Arbeitsgänge sowie der Kapazitätsbeschränkung der erneuerbaren Ressourcen.\footnote{Vgl. \cite{demeulemeester2011robust}, S. 23} Die im folgenden aufgestellte Zielfunktion des RCPSP zur Minimierung der Projektdauer ist die gängige Version der Kapazitätsplanung,\footnote{Vgl. \cite{drexl1997neuere}, S. 98} andere Variationen sind aber ebenfalls möglich.\footnote{Vgl. \cite{talbot1982resource}, S. 1200}\\

Nachfolgend wird das deterministische RCPSP in diskreter Zeit formuliert.\footnote{????} Charakteristisch für eine mathematische Modellformulierung in diskreter Zeit sind die Zeiteinheiten, die den Perioden $t, \tau$ entsprechen.\\

\textbf{Modell RCPSP}
\begin{eqnarray} \label{Ziel}
\min Z = \sum_{t=FE_{I}}^{SE_{I}}t \cdot X_{I,t}\hfill  
\end{eqnarray}

unter Beachtung der Restriktionen
\begin{multline} \label{N1}
\sum_{t=FE_{i}}^{SE_{i}} X_{it} = 1
\hfill   i = 1,...,I
\end{multline}\vspace{-3.0ex}

\begin{multline} \label{N2}
\sum_{t=FE_{h}}^{SE_{h}}t \cdot X_{ht} \leq \sum_{t=FE_{i}}^{SE_{i}}(t - d_{i}) \cdot X_{it}
\hfill   i =1,...,I;\; h \in \mathcal{V}_{i}
\end{multline}\vspace{-3.0ex}

\begin{multline} \label{N3}
\sum_{i=1}^{I}\sum_{\tau=\max(t,FE_{i})}^{\tau=\min(t+d_i-1,SE_i)}k_ {ir} \cdot X_{i\tau} \leq kp_{r}
\hfill   r =1,...,R;\; t=1,...,T
\end{multline}\vspace{-3.0ex}
\begin{multline} \label{N4}
X_{it} \in \{0,1\}
\hfill   i \in \mathcal{I};\; t \in \{FE_{i},...,SE_{i}\}\end{multline}\vspace{-6.0ex}\\

Es wird ein Projekt betrachtet, dass aus $I$ unterschiedlichen Arbeitsgängen besteht. Jeder Arbeitsgang $i$ hat eine definierte Menge von zu erledigenden Vorgängerarbeitsgängen $h \in \mathcal{V}_{i}$. Des Weiteren ist für die Fertigstellung des Projekts die Abarbeitung der Arbeitsgänge in topologischer Reihenfolge notwendig. D. h. der Vorgänger $h$ hat stets eine kleinere Ordnungszahl als sein Nachfolger $i\;(h<i)$ und muss zur Fortsetzung des Projektverlaufs beendet sein. Die Bearbeitungsdauer eines Arbeitsgangs $i$ wird mit dem Parameter $d_{i}$ festgelegt.  Bei dem RCPSP in diskreter Zeit wird die Annahme getroffen, dass die Dauer durch einen ganzzahligen Parameter abgebildet wird. Der Startzeitpunkt des Projekts ist $t = 0$ und erstreckt sich über einen Gesamtzeitraum von $T$ Perioden. Um die Reihenfolgebedingungen einzuhalten, werden einem Projekt zwei Dummy-Arbeitsgänge \glqq Beginn\grqq\;($i=1$) und \glqq Ende\grqq\;($i=I$) hinzugefügt, welche mit einer Dauer von $0$ Zeiteinheiten bewertet werden.\footnote{Vgl. \cite{zimmermann2006projektplanung}, S. 66} Dadurch wird der Projektbeginn und das Projektende exakt terminiert. Der Parameter $k_{ir}$ stellt die benötigten Kapazitäten der erneuerbaren Ressource $r$ bei der Durchführung von Arbeitsgang $i$ dar. Die Ressourcen $r \in R$ sind in einer Periode innerhalb des Umfangs ihrer Kapazität $kp_{r}$ nutzbar. Da es sich um erneuerbare Ressourcen handelt, stehen diese zu jeder neuen Periode in vollem Umfang erneut zur Verfügung. Ungenutzte Ressourcen sind jedoch nicht auf nachfolgende Arbeitsgänge und Perioden übertragbar.\footnote{Vgl. \cite{kellenbrink2014einfuhrung}, S. 12} Um den Fertigstellungszeitpunkt der einzelnen Arbeitsgänge $i$ festlegen zu können, wird der Modellformulierung in diskreter Zeit die binäre Entscheidungsvariable $X_{it}$ hinzugefügt.\footnote{Vgl. \cite{pritsker1969multiproject}, S. 94} Diese Binärvariable nimmt den Wert $1$ an, falls der Arbeitsgang $i$ zum Zeitpunkt $t$ beendet wird.\\

Mittels der Zielfunktion \eqref{Ziel} wird der Fertigstellungszeitpunkt des Projekts minimiert. Dafür wird der Zeitraum zwischen dem frühesten und spätesten Fertigstellungszeitpunkt $FE_{I}$ und $SE_{I}$ aller durchzuführenden Arbeitsgänge $I$ betrachtet. Nebenbedingung \eqref{N1} stellt sicher, dass ein Arbeitsgang $i$ zwischen dem jeweiligen für diesen Arbeitsgang geltenden frühesten und spätesten Fertigstellungszeitpunkt nur exakt ein Mal durchgeführt wird. Die Reihenfolgerestriktion wird mit der Nebenbedingung \eqref{N2} eingehalten. Sie stellt sicher, dass jeder Vorgänger $h \in \mathcal{V}_{i}$ beendet ist, bevor der Arbeitsgang $i$ startet. Der Term $(t - d_{i})$ garantiert für den Arbeitsgang $i$, dass dieser erst beginnt, sobald der Vorgänger $h$ mit der Dauer $d_{i}$ abgeschlossen ist.
Der Parameter $kp_{r}$ spiegelt die Kapazitätsgrenze für eine erneuerbare Ressource $r$ je Periode $t$ wieder. In Nebenbedingung \eqref{N3} findet zum einen eine formale Darstellung dieser Kapaiztätsbegrenzung statt. Zum anderen wird der Ressourcenverzehr während der gesamten Bearbeitungsdauer der Fertigstellung beachtet, in dem der Kapazitätsbedarf $k_{ir}$ aller Arbeitsgänge $I$ summiert wird. Eben diese Summe wird schließlich durch $kp_{r}$ beschränkt. %????Da ist irgendwas falsch...
Mit der Nebenbedingung \eqref{N4} wird die Binärvariable $X_{it}$ für den Zeitraum $t = \{FE_{i},...,SE_{i}\}$ formal definiert. Aufgrund der Reihenfolgebeziehung \eqref{N2} darf der jeweils betrachtete Arbeitsgang nur in diesem Zeitraum fertiggestellt werden.
Die gemischt-ganzzahlige Modellformulierung lässt sich durch Standard-Lösungsverfahren exakt lösen.\footnote{z. B. mittels eines Branch-and-Bound-Verfahrens, Vgl. \cite{kellenbrink2014einfuhrung}, S. 14}

\subsection{Kostenplanung}
Aufbauen auf der Kapazitätsplanung kann das RCPSP um die Nutzung von Zusatzkapazitäten der Ressourcen erweitert werden, damit dem Projektplanungsmodell gestattet ist den Vorgänge zusätzliche Kapazitätseinheiten der notwendigen Ressourcen bereitzustellen. Die Kapazitätsrestriktion wird dementsprechend um die Entscheidungsvariable $O_{rt} \geq 0$ erweitert. Die Variable $O_{rt}$ beschreibt die Einheiten an Zusatzkapazitäten einer Ressource $r$ in der Periode $t$. Damit steht nicht die Einhaltung der verfügbaren Kapazitäten im Vordergrund, sonder unter Beachtung der Projektstruktur die aufgewendeten Zusatzkosten des Projekts. Dem Optimierungsmodell ist es damit gestattet durch Erhöhung der Kapazitäten der Ressourcen die anfängliche Ressourcenbeschränkung zu umgehen. Bei der Modellerweiterung der Kostenplanung wird der Parameter $oc_r$ eingeführt, der für eine betrachtete Ressource $r$ die Kosten einer Einheit der Zusatzkapazitäten beschreibt. Ziel des Optimierungsmodells ist es damit die Kosten des Projekts zu minimieren. Das Modell hilft damit der Entscheidung, ob durch Einführen der Möglichkeit von Zusatzkapazitäten das Projektziel verbessert erreicht wird. Es handelt sich um den Trade-off des frühzeitigen Erreichens des Projektziels durch Nutzung von Zusatzkapazitäten und der gesamten Projektkosten die für das Projekt aufgewendet werden sollen.\\

Nachfolgend wird das deterministische RCPSP+ in diskreter Zeit formuliert.\\

\textbf{Modell RCPSP+}
\begin{eqnarray} \label{Ziel2}
\min Z = \sum_{t=1}^{T}\sum_{t=1}^{R} oc_{r} \cdot O_{r,t}\hfill  
\end{eqnarray}

unter Beachtung der Restriktionen \eqref{N1}, \eqref{N2}, \eqref{N4} sowie
\begin{multline} \label{N5}
\sum_{i=1}^{I}\sum_{\tau=\max(t,FE_{i})}^{\tau=\min(t+d_i-1,SE_i)}k_ {ir} \cdot X_{i\tau} \leq kp_{r} + O_{rt}
\hfill   r =1,...,R;\; t=1,...,T
\end{multline}\vspace{-3.0ex}
\begin{multline} \label{N6}
O_{rt} \geq 0
\hfill   r =1,...,R;\; t=1,...,T \end{multline}\vspace{-6.0ex}\\

Bei dem RCPSP+ wird die Zielfunktion insoweit formuliert, dass über alle Perioden $t\in T$ und über alle Ressourcen $r\in R$ die Summe der Kosten $oc_r$ für die Anzahl an notwendige Einheiten an Zusatzkapazität $O_{rt}$ minimiert wird. Weiterhin bleibt die Nebenbedienung \eqref{N1} und \eqref{N2} bestehen, dass jeder Vorgang exakt einmal zwischen dem frühesten Ende ($FE_i$) und dem spätesten ($SE_i$) fertiggestellt und die Topologie der Vorgänge eingehalten wird. Weiterhin gilt die Nebenbedingung \eqref{N4}, dass es sich bei der Entscheidungsvariable $X_{jt}$ um eine binäre Variable handelt. Erweitert wird das RCPSP aus der Kapazitätsplanung mit einer modifizierten Nebenbedingung zur Einhaltung Kapazitätsbeschränkung. Mit der Nebenbedingung \eqref{N5} wird die Kapazitätsrestriktion für eine Ressource $r\in R$ in einer Periode $t\in T$ eingehalten, jedoch ist es dem Modell gestattet die vorhandene Ressourcenkapazität $kp_r$ um die Ausprägung der Entscheidungsvariable $O_{rt}$ zu erweitern. Durch Lösen des Optimierungsmodells wird der Ablaufplan des Projekts unter Beachtung der unterschiedlich zulässigen Gesamtdauern $SE_I$ generiert. Weiter wird für jede Ressource $r\in R$ zur jeweiligen Periode $t\in T$ die notwendige Anzahl an benötigter Zusatzkapazität $O_{rt}$ ermittelt. Die Nebenbedingung \eqref{N6} beschreibt die Eigenschaft der Entscheidungsvariable $O_{rt}$, dass es sich um eine positive Variable bzw. einen Nullwert handelt.

\subsection{Ruby on Rails}

Das Frameworks Ruby on Rails (RoR) zur Entwicklung von Web-Applikationen mit Datenbankbezug wurde von David Heinemeier Hansson im Jahre 2004 erstmals vorgestellt.\footnote{Vgl. \cite{ruby2004}} Mit dem Name von RoR wird klar, das das Framework die Programmiersprache Ruby nutzt. Ruby wird von den den meisten gängigen Betriebssystem unterstützt (Microsoft Windows, Apple Mac OS X, Linux, etc.) und ist bspw. dem Betriebssystem Apple Mac OS X in der Version 1.8.7 standardmäßig integriert.\footnote{Vgl. \cite{ruby-schienen}} Bei Ruby handelt es sich um eine objekt-orientierte Programmiersprache mit dem Grundsatz \textit{principle of least surprise} und folgt einigen Besonderheiten, wie z. B. einer einfachen Sprachsyntax, keiner typisierten Variablen und einer reinen Objektorientierung.\footnote{Vgl. \cite{Walter:2008aa}, S. 297-298} Abbildung \ref{Terminal} zeigt das Terminal von Apple Mac OS X mit typischen Ruby Kommandobefehlen. RoR nutzt diesen einfachen Syntax zur Entwicklung von Web-Applikationen, wobei aufgrund einfacherer Bedienung auf integrierte Entwicklungsumgebung zurückgegriffen wird, wie z. B. RadRails oder RubyMine.\footnote{Vgl. \cite{hartl2012ruby}, S. 10} \\

\begin{figure}[h!]
  \begin{center}
    \includegraphics[width=120mm]{Bilder/Terminal.pdf}
    \caption{Terminalfenster unter Apple Mac OS X}  \label{Terminal}
  \end{center}
\end{figure}

Mit Hilfe des RoR Frameworks lassen sich dadurch schnell Web-Applikationen mit Datenbankbezug entwicklen, wobei der wesentlichen Vorteile in der Softwarearchitektur des Model-View-Controller-Paradigmas liegt.\footnote{Vgl. \cite{walter2008ruby}, S. 463} Das Paradigma besagt, dass eine durch einen Browser angestoßene Anfrage an den Server durch den Rails \texttt{controller} verarbeitet wird. Der \texttt{controller} verarbeitet die Anfrage und leitet die nachfolgenden Schritte ein. Bei Web-Applikationen erfolgt eine solche Verarbeitung durch anzeigen bzw. dem sogenannten \textit{rendern} von HTLM-Dokumenten der Rails \texttt{views} , die von Browsern angezeigt werden können. Der \texttt{controller} rendert die \texttt{views} und ermöglicht weitere RoR-Befehle im HTML-Dokument. Bei komplexen und dynamischen Seiten übernimmt der \texttt{controller} geforderte Daten aus den Rails \texttt{models}, die wiederum mit einer Datenbank verbunden sind. Durch diese Architektur lassen sind umfangreiche und an spezifische Anfrage angepasste Web-Applikationen entwickeln. Ein weiterer Vorteil von RoR ist die einfache Implementierung von Unterprogrammen. Ein Unterprogramme ist in Ruby/RoR ein \texttt{Gemfile}, das durch den Bundler zur bestehenden Web-Applikation hinzugefügt wird.\footnote{Vgl. \cite{hartl2012ruby}, S. 9-17} Im nachfolgenden Kapitel wird die Entwicklung einer Web-Applikation mittels RoR beschrieben. Dabei liegt die Besonderheit der Ausarbeitung auf Integration eines notwendigen Unterprogramms (\texttt{Gemfile}) und die Verbindung zum Programm GAMS, damit das in diesem Kapitel vorgestellte Projektplanungsprobem gelöst werden kann.


\section{Implementierung des RCPSP mittels Ruby on Rails} \label{Haupt}

\subsection{Darstellung der Funktionsweise der Anwendung anhand eines Userguides}\label{User}
Die Funktionsweise der mit \textit{RoR} programmierten Anwendung \textit {\glqq Projektplanung\grqq} zur Lösung der Kapizitäts- und Kostenplanung des \textit{RCPSP} lässt sich am anschaulichsten mit Hilfe eines Userguides darstellen. Die Applikation kann über nachfolgenden Terminalbefehl auf ein lokales Computerverzeichnis geklont werden.
\begin{lstlisting}[style=Befehl]
$ git clone https://github.com/rb4k/as-rcpsp.git
\end{lstlisting}

Neben der Besonderheiten, die durch das Problem der Projektplanung auftreten, können im selben Zuge auch die Spezifika der einzelnen Benutzerrollen aufgezeigt werden. Beachtet werden muss, dass die hier vorgestellte Web-Applikation auf der Arbeit von \cite{hartl2012ruby} aufbaut.\\

Zunächst wird die Anwendung aus der Sicht eines Anwenders betrachtet, der sich nicht in die Applikation per Benutzererkennung eingeloggt hat. Konkret kann man sich darunter einen potentiellen Mitarbeiter des entsprechenden Projektes vorstellen, der sich über die Projektplanung informieren möchte, um sich gegebenenfalls als Mitarbeiter im Projekt (User) anzumelden. Im Testbetrieb wird der Ruby-Servers gestartet und durch Eingabe der URL \textit {"http://localhost:3000/"} in die Adresszeile eines beliebigen modernen Browers wird die Startseite der Projektplanung angezeigt (siehe Abbildung \ref{Start}). Alternativ ist der Betrieb auf einem Webserver möglich, sofern die benötigte Software installiert und betriebsbereit ist. Auf der Startseite hat der User zum einen die Möglichkeit, sich anzumelden bzw. sich einzuloggen, für den Fall, dass er bereits User der Anwendung ist. \\

\begin{figure}[h!]
  \begin{center}
    \includegraphics[width=150mm]{Bilder/Startseite_unsigned.png}
    \caption{Startseite Projektplanung Applikation}  \label{Start}
  \end{center}
\end{figure}

Bei der Startseite (\texttt{home.html.rb}) der RoR-Applikation handelt es sich um eine statische Seite (\texttt{static\_pages}) der \texttt{views}. Weiter gehören zu dieser Kategorie der HTML.RB-Dokumente die Seiten \texttt{about}, \texttt{contact}, \texttt{help} und \texttt{rcpsp}. Letztere wird zum späteren Zeitpunkt thematisiert. Ein Beispiel einer statischen Seite eines RoR \texttt{views} liefert Quellcode \ref{test} im Anhang \ref{rcodes}.\\

Anhand der \texttt{static\_pages} kann die Besonderheit von RoR deutlich gemacht werden. Durch das Model-View-Controller-Paradigma hilft der \texttt{static\_pages\_controller} bei der Verarbeitung von Anfragen. Es handelt sich hier um das typische ein Scaffolding (Bauprinzip) in RoR, bei dem ein \texttt{controller}, \texttt{models} und \texttt{views} erstellt werden.\footnote{Vgl. \cite{walter2008ruby}, S. 464} Generiert werden können die Scaffolds durch  einen Ruby-Befehl im Terminalfenster.
\begin{lstlisting}[style=Befehl]
$ rails generate scaffold <name> <name:datentyp> 
\end{lstlisting}

Wie der Name aber schon andeutet, bedarf es bei den statischen Seiten kaum der Verarbeitung von Datensätzen der RoR \texttt{models} zur Erstellung von dynamischen Seiten, wie der Quellcode \ref{spc} im Anhang \ref{rcodes} zeigt.\\

Für die \texttt{static\_pages} bedarf es einen speziellen \textit{Match}, der in \texttt{config/routes.rb} Datei hinterlegt wird (Vgl. Quellcode \ref{routes}). Die \texttt{config/routes.rb} ordnet den Scaffolds und HMTL-Dokumente spezifische Verzeichnisse in der Applikation zu. RoR erkennt die Unterseiten der angelegten Scaffolds und ermöglicht die Verlinkung der Seiten auch ohne spezifische Angabe (Vgl. Quellcode \ref{routes} im Anhang \ref{rcodes}).\\

Mit dem Link \textit{Anmelden} erfolgt die Weiterleitung von der Startseite zur Anmeldeseite. Beschließt sich der Besucher der Seite, sich für das Projekt anzumelden, muss er alle Felder des Anmeldeformulars befüllen.

\begin{figure}[h!]
  \begin{center}
    \includegraphics[width=120mm]{Bilder/Anmeldung.png}
    \caption{Anmeldebildschirm}  \label{Anm}
  \end{center}
\end{figure}

Neben dem Namen, einer Mailadresse und eines konformen Passwortes sind projektspezifische Informationen zur erfolgreichen Registrierung nötig. Im Feld \textit{Arbeitszeit pro Tag} muss ein entsprechender Wert eingegeben werden, den der neue User bereit ist, pro Tag für das Projekt an Zeit zu investieren. Wird in diesem Feld keine ganze Zahl, sondern eine Dezimalzahl oder ein Wort eingegeben, kann die Anmeldung im System nicht stattfinden. Es wird ein Fehler angezeigt, der das Defizit aufzeigt und behoben werden muss (siehe Abbildung \ref{Fehler}). Ausgelöst wird dieser Fehler durch einen Vermerk im zugehörigen RoR \texttt{models}, dass es sich um eine ganzahlilge Zahl handelt (\textit{Integer}). Der Quellcode \ref{fehler_code} im Anhang \ref{rcotes} zeigt dies anhand des hier betrachteten Beispiels.\\

Auf der Startseite (sowie allen anderen Seiten) sind eben Links wie \textit{Hilfe} und \textit{Kontakt} auch der Link zu \textit{Ressourcen}-Übersicht, in der alle Ressourcen des Projektes gelistet sind. Hier kommt der Grundsatz von RoR zum Tragen: \textit{Don’t repeat yourself} Der Gestaltung und der Aufbau einer jeden Seite in der Web-Applikation orientiert sich anhand der CSS-Stylesheets bzw. der Layout-Dateien. Die Layout-Dateien sind unter \texttt{app/views/} gelistet und definieren auf jeder Seite spezifische Bereiche. Die \texttt{application.html.erb} generiert für jede Seite dieses einheitliche Layout, unterstützt durch die Dateien \texttt{\_footer.html.erb} und \texttt{\_header.html.erb}. Im \texttt{\_header.html.erb} ist der Link zur Ressourcen-Übersicht vermerkt (Vgl. Quellcode \ref{header} im Anhang \ref{header}).\\

Der \texttt{\_header.html.erb} zeigt schon einige \textit{If}-Befehle, mit denen unterschiedliche Daten anhand der Eigenschaften unangemeldeten, angemeldeten und Admin-Usern angezeigt werden. Durch Folgen des Links \textit{Ressource} wird die Index-Seite des der RoR \texttt{views/ressources} angezeigt (Siehe Abbildung ???). Der Quellcode \ref{index_res} im Anhang \ref{rcodes} zeigt die notwendige Programmierung für die Seite.\\

RoR durchläuft aufgrund der Aktivierung des Links die Aktion \textit{index} des dazugehörigen Controllers \texttt{resources\_controller.rb} und generiert die zugehörige HTML-Seite (\texttt{views}). Die Indexseite prüft, ob der aktuelle User angemeldet ist. Abhängig dieser Entscheidung integriert RoR einen unterschiedlichen Seiteninhalt. Sofern der aktuelle User nicht angemeldet ist, wird eine vereinfachte Ressourcen-Übersicht angezeigt (siehe Abbildung \ref{Anm}). In dieser Ansicht sind alle jeweilig aktuellen Ressourcen mit zugehörigen Namen aufgelistet, sowie dem Link \textit{Bewerben}, der wiederum mit der Anmeldeseite verlinkt ist.\\

Findet keine Anmeldung in die Web-Applikation statt, sind keine weiterführenden Tätigkeiten möglich. Die Startseite liefert keine weiterführenden Informationen und bei der Eingabe von anderen Links in die Adresszeile des Browers wird der aktuelle User zur \textit{Login}-Seite geführt, da alle Daten für nicht angemeldete Anwender gesperrt sind. %Die Verlinkung \textit{"http://localhost:3000/rcpsp/"} öffnet zwar die Seite der Projektplanung, alle angezeigten Buttons führen aber ebenfalls direkt zur Anmeldemaske. \\
Um die Applikation nutzen zu können, ist demzufolge die Anmeldung als User zwingend notwendig. Findet diese entweder nach erstmaliger Registrierung über den Link \textit{Anmelden} oder über \textit{Login} statt, wird die eigene Profilseite angezeigt (siehe Abbildung \ref{Profil}).\\

\begin{figure}[h!]
  \begin{center}
    \includegraphics[width=120mm]{Bilder/Profilseite.png}
    \caption{Profilseite eines Users}  \label{Profil}
  \end{center}
\end{figure}

Die Profilseite gibt einen Überblick über all die Daten, die für den User in Hinblick auf das Projekt relevant sind. Es werden die Daten dargestellt, die bei der Anmeldung angegeben wurden (Arbeitszeit, Rolle im Projekt) sowie die Vorgänge, die durch die Wahl der Ressource für diesen User relevant sind, in denen er also arbeiten muss. Der Aufbau der Seite ist im Quellcode \ref{show_user} im Anhang \ref{rcodes} dargestellt.\\

Zu jedem Vorgang wird die Dauer und gegebenenfalls die Zeitspanne angegeben, wann er jeweils stattfindet. Die Grenze liegt zwischen dem frühesten Startzeitpunkt $FA_{i}$ und spätesten Endzeitpunkt $SE_{i}$ des Vorgangs $i$. Ebenfalls wird der kritische Pfad angezeigt. Dieser zeigt für die aufgeführten Vorgängen den Endzeitpunkt nach Start des Projekts unter Einhaltung der Ressourcenbeschränkung an, jeweils in Zeiteinheiten. Ob diese Tabelle mit Daten gefüllt ist, hängt davon ab, ob das Kapazitäts- bzw. Kostenplanungsproblem bereits gelöst wurde. Möchte der User seine Daten, wie z.B. die Wahl der Ressource oder die Quantität der Arbeitszeit, ändern, gelangt er über den Button \textit{Einstellung} zu einer Seite, die äquivalent aufgebaut ist wie die Anmeldeseite, um dort seine Daten zu aktualisieren. Nach korrekter Eingabe können die Daten über den Button \textit{Speichere Änderungen} gesichert werden. Auf der Profilseite erscheint daraufhin eine Anzeige \textit{Profil updated} mit der Bestätigung, dass das Profil aktualisiert wurde. Im Vergleich zum fremden Anwender gelangt der angemeldete User außerdem in der Kopfzeile über den Link \textit{Mitarbeiter} über eine Übersicht aller Mitarbeiter, die für das Projekt auf dieser Applikation angemeldet sind. Die Profilseite jedes Mitarbeiters kann betrachtet werden mit all den Informationen, die auch auf der eigenen Profilseite einzusehen sind. Es können jedoch keine Änderungen vorgenommen werden. Neben der Verlinkung zu der Übersicht der Mitarbeiter lässt sich in der Kopfzeile ein Feld \textit{Menü} finden, dass die Unterpunkte \textit{Profil}, \textit{Einstellungen} und \textit{Logout} enthält. Die Verlinkung \textit{Profil} stellt eine Verlinkung zur Profilseite dar, unter \textit{Einstellungen} kann das eigene Profil aktualisiert werden.\\

Unter \textit{Ressourcen} kann der User, wie auch der nicht angemeldete Anwender, zur Übersicht der vorhandenen Ressourcen gelangen. Die Anzeige stellt sich für den angemeldeten User jedoch vielfältiger dar, als für den einfachen Anwender (siehe Abbildung \ref{ResUs}), da hier eine andere Quellcode integriert wird (Vgl. Quellcode \ref{index_res}). Der Quellcode \ref{sign_res} im Anhang \ref{rcodes} zeigt den integrierten Inhalt.\\

\begin{figure}[h!]
  \begin{center}
    \includegraphics[width=120mm]{Bilder/Ressourcen_User.png}
    \caption{Übersicht der Ressourcen für User}  \label{ResUs}
  \end{center}
\end{figure}

Für den User sind alle Eigenschaften der verschiedenen Ressourcen einsehbar. Es werden die Gesamtkapazität, Kosten pro ME, Grundkosten und Zusatzkosten pro ME angezeigt. Wurde bereits eine Lösung für das Problem der Kostenplanung ermittelt, werden die kalkulierten Werte für die Zusatzeinheiten, gesamten Zusatzkosten und die Gesamtkosten pro Ressource dargestellt. Zudem wird der Zielfunktionswert, bei der Kostenplanung die gesamten anfallenden Zusatzkosten, in Verbindung mit dem Zeitpunkt der Optimierung über der Tabelle dargestellt. Die Darstellung der Tabellen in dieser Web-Applikation orientiert sich an dem Bootstrap-Framework\footnote{http://getbootstrap.com}. Alternativ bietet die App die Anzeige der Tabellen anhand einer JavaScript-Tabelle.\footnote{Auf Implementierung wurde jedoch aufgrund der möglichen Inkompatibilität zu bestimmten Browser und aufgrund der Laufzeitverbesserung der Web-Applikation verzichtet.} Über den Button \textit{Anzeigen} in der hier betrachteten Tabelle sind die Eigenschaften einer Ressource separat einsehbar. Da der User bzw. Mitarbeiter in diesem Modell durch die Planung innerhalb des Projektes eingeteilt wird und seine Rechte nicht über die Organisation der eigenen Daten hinaus reicht, hat er keine weiteren Kompetenzen bei der Nutzung dieser Applikation. \\

Die Verwaltung der Mitarbeiter und die Organisation sowie Durchführung der Projektplanung kann ausschließlich nach der Anmeldung als Administrator erfolgen. Der Admin gilt in dieser Anwendung als durchführende Gewalt. In dieser Testsituation ist \textit{"Example User"} mit den dafür notwendigen Berechtigungen ausgestattet. Alternativ lässt sich durch Änderung der booleschen Variable \texttt{admin = true} der Datenbank zum RoR \texttt{models/users} die Eigenschaft auch auf andere Datensätze (User) übertragen. Nach der erfolgreichen Anmeldung als User mit administrativen Rechten erscheint zunächst erneut die Profilseite, sofern die Anmeldung über die Startseite erfolgt. Im Gegensatz zu normalen Usern bietet die Seite eines Admins jedoch zusätzliche Handlungsspielräume neben der einfachen Auflistung der Vorgänge (siehe Abbildung \ref{ProAd}). Er hat die Möglichkeit, die Dauer der Vorgänge abzuändern oder Vorgänge aus dem Projekt zu löschen. Dies wird wieder über einen \textit{If}-Befehl gesteuert, wie der Quellcode \ref{show_user} aus dem Anhang \ref{rcodes} zeigt.\\

\begin{figure}[h!]
  \begin{center}
    \includegraphics[width=120mm]{Bilder/Profilseite_Admin.png}
    \caption{Profilseite des Administrators}  \label{ProAd}
  \end{center}
\end{figure}       

In gleicher Weise stellt sich die Ausweitung der Kompetenzen bei den Ressourcen dar. Über die Auswahl des Links \textit{Ressourcen} über den \textit{Header} wird zur Ressourcenübersicht verbunden und dort können nun die Ressourcen ebenfalls gelöscht oder die Eigenschaften (Kosten und Zusatzkosten je ME) verändert werden. Des Weiteren kann über den Button \textit{Neue Ressource anlegen} eben dies vollzogen werden (Vgl. Quellcode \ref{index_res} im Anhang \ref{rcodes}).\\      

\begin{figure}[h!]
  \begin{center}
    \includegraphics[width=120mm]{Bilder/Ressourcen_Admin.png}
    \caption{Übersicht der Ressourcen aus Sicht des Administrators}  \label{ResAd}
  \end{center}
\end{figure}     
  
Um nun zur Kernaufgabe der Applikation, der Projektplanung, zu gelangen, kann entweder der Button \textit{Zurück zur Projektplanung} auf der Seite zur Ressourcenübersicht getätigt werden, oder ausgehend von jeder beliebigen Seite der Web-Applikation in der Kopfzeile (\textit{Header}) wird unter \textit{Menü} der Unterpunkt \textit{Projektplanung} ausgewählt (siehe Abbildung \ref{RCPSP}). Bei dieser statischen Seite fließen Daten aus dem RoR \texttt{models/project} ein. Bei diesem Modell handelt es sich um eine Hilfsdatenbank ohne weiterer Beziehung zu anderen Modellen (Vgl. Abbildung \ref{schema} im Anhang \ref{db-schema}). Sie fungiert als Datenbank für unabhängige Parameter und hat damit nur einen Datensatz. Der Controller der statischen Seiten ruft über die Aktion \texttt{rcpsp} diesen Datensatz auf (Vgl. Quellcode \ref{spc} aus Anhang \ref{rcodes}). Dadurch kann das HTML.RB-Dokument \texttt{views/static\_pages/} den Datensatz aufgreifen und Formularfelder zur Eingabe der unabhängigen Parameter bereitstelle. Zu den unabhängigen Parametern dieses Formulars zählen die Datenfelder \texttt{path}, \texttt{startdate} und \texttt{deadline}, auf die im Verlauf der weiteren Beschreibung der Web-Applikation näher eingegangen wird.\\ %Wie bereits erwähnt, hat nur der Admin die Berechtigung, dieses Menü aufzurufen.


\begin{figure}[h!]
  \begin{center}
    \includegraphics[width=120mm]{Bilder/Projektplanung.png}
    \caption{Projektplanung mit dem RCPSP - Übersicht}  \label{RCPSP}
  \end{center}
\end{figure}

Im oberen Bereich der Seite sind die Verlinkungen zur Verwaltung der nötigen Inputs zur Lösung beider Planungsproblematiken angesiedelt. Neben den bereits behandelten Links zu den Vorgängen und Ressourcen finden sich Verlinkungen zu den Vorgangsrelationen und Vorgang-Ressourcen-Kombinationen.\\ 

Die Übersicht der Relationen zwischen den Vorgängen stellt eine Auflistung eines jeden Vorgänger und Nachfolger dar. Ein Admin kann diese Relationen löschen oder neue anlegen. Wenn er sich dazu entschließt, eine neue anzulegen, ist zu beachten, dass ein Strukturplan eines Projektes keine Zyklen beinhalten darf. Damit Zyklen verhindert werden, findet beim Prozess des Anlegens einer neuen Vorgangsrelation eine Prüfung statt. Beinhaltet die neu angelegte Relation einen Zyklus, tritt ein Fehler auf und die Relationen muss überarbeitet werden (siehe Abbildung \ref{VorErr}). So wird verhindert, dass der Strukturplan Zyklen enthält. Dieser Vorgang wird gesteuert durch die dafür zuständige Aktion \texttt{create} aus dem RoR \texttt{controllers/procedure\_procedures\_controller.rb} (Vgl. Quellcode \ref{ppc} im Anhang \ref{rcodes}). Inbegriffen in die Aktion ist ein frei-verfügbares Unterprogramm (\texttt{Gem})\footnote{https://github.com/monora/rgl}. In Kapitel \ref{rgl} wird die Integration und Funktionsweise dieses Unterprogramms beschrieben. \\

\begin{figure}[h!]
  \begin{center}
    \includegraphics[width=120mm]{Bilder/Vorgangsrel_Fehler.png}
    \caption{Fehler aufgrund eines Zyklus in der topologischen Reihenfolge}  \label{VorErr}
  \end{center}
\end{figure}
 
Der Button \textit{Vorgang-Ressourcen-Kombination} führt zu einer Übersicht der verschiedenen Ressourcen zu den Vorgängen. Neben der Auflistung können die Kombinationen verändert, gelöscht oder neu erstellt werden. Bei der Veränderung oder Erstellung ist zu beachten, dass die Angabe des Kapazitätsbedarfs nur mit Hilfe einer ganzen Zahl erfolgen darf. Entsprechend der Kapazitätsangabe bei der Bearbeitung eines Profils erscheint bei jeder anderen Art von Eingabe ein Fehler, der die Datenspeicherung verhindert (Vgl. Quellcode \ref{prm} im Anhang \ref{rcodes}).\\

Nachdem all diese Daten geprüft und gegebenenfalls verändert wurden, steht die Basis sowohl für die Optimierung der Kapazitäts- als auch der Kostenplanung. Bevor der Optimierungsprozess stattfinden kann, müssen noch einige Rahmenbedingungen geprüft werden. Da die Optimierung mit dem Programm \textit{GAMS} stattfindet, muss die Applikation auf dieses Programm zurückgreifen können. Dafür muss \textit{GAMS} auf dem hiesigen Computer installiert sein. Nach der Recherche des Installationsortes muss der korrekte Pfad in das dafür vorgesehene Feld der Übersichtsseite zur Projektplanung eingetragen werden, in dem der Beispielpfad zu sehen ist. Nach der Eingabe wird der Pfadzugriff durch \textit{Verzeichnis aktualisieren} in der Datenbank des RoR \texttt{models/project} gesichert (siehe Abbildung \ref{RCPSP}). Neben den Programmpfad muss ein Termin ausgewählt werden, zu dem das Projekt startet (siehe Abbildung \ref{Startdatum}). Anhand dieses Startdatums werden alle Daten bezüglich der Vorgänge berechnet, zudem stellt der Starttermin bei der Kostenplanung einen wichtigen Faktor dar. Durch die Betätigung des Feldes, das ein Muster anzeigt, öffnet sich ein Kalendermenü, in dem ein beliebiges Datum ausgewählt werden kann. Bei dem Datumsfeld handelt es sich ebenfalls um eine Applikationserweiterung (\texttt{Gem}) namens \glqq Bootstrap-Datepicker-Rails\grqq\footnote{https://github.com/Nerian/bootstrap-datepicker-rails} (Vgl. Quellcode \ref{gemfilie} im Anhang \ref{rcodes}). Es handelt sich hier um eine Unterprogramm inkl. dazugehöriger JavaScrip-Datei. \\  

\begin{figure}[h!]
  \begin{center}
    \includegraphics[width=120mm]{Bilder/Projektplanung_Datum.png}
    \caption{Einstellung des Starttermins anhand eines Kalendermenüs}  \label{Startdatum}
  \end{center}
\end{figure}

Nachdem das \textit{GAMS}-Verzeichnis und der Starttermin eingestellt wurden, kann die Kapazitätsplanung durch die Betätigung des Buttons \textit{Optimiere Kapazitätsplanung} durchgeführt werden. Es handelt sich hier um die Aktion \texttt{optimize} des RoR \texttt{rcpsps\_controller} (Vgl. Quellcode \ref{rcpspc} im Anhang \ref{rcodes}). Die Aktion dient dazu die Include-Dateien für die \textit{GAMS}-Optimierung zu schreiben und eben diese durch Aufrufen der \textit{GAMS}-Software zu starten. Bei der \textit{GAMS}-Optimierung handelt es sich um die Datei mit dem Quellcode \ref{rcpsp1} aus Anhang \ref{Imp}. Nach einer Rechenzeit, währenddessen der Button, mit dem die Optimierung gestartet wurde, auf den Rechenprozess hinweist, leitet die Applikation den Admin direkt zu der Übersicht der Vorgänge. Dieser Schritt wird durch die Hilfsaktion \texttt{solution} des RoR \texttt{rcpsps\_controller} unterstütz, die parallel aufgerufen wird. Nachdem die \textit{GAMS}-Optimierung vollzogen ist, liest der RoR \texttt{rcpsps\_controller} die von der \textit{GAMS}-Optimierung erstellten Text-Dateien ein und schreibt diese direkt in die dafür vorgesehen Datenbank. Bei der Kapazitätsplanung wird hauptsächlich die Datenbank des RoR \texttt{models/procedure} angesprochen. Wie bereits in Abbildung \ref{ProAd} dargestellt, sind in der Übersicht der Vorgänge alle möglichen Zeitpunkte dargestellt, an denen die einzelnen Vorgänge stattfinden können.  Zusätzlich wird die Projektdauer über der Tabelle und der kritische Pfad, durch den die Projektdauer erzielt wird, in die Tabelle geschrieben (siehe Abbildung \ref{Kap}).
\begin{figure}[h!]
  \begin{center}
    \includegraphics[width=120mm]{Bilder/Kapazitaetsplanung.png}
    \caption{Ergebnis der Kapazitätsplanung}  \label{Kap}
  \end{center}
\end{figure}

Die Besonderheit der Aktion \texttt{optimize} ist, dass RoR die durch die \textit{GAMS}-Optimierung (Quellcode \ref{rcpsp2} aus Anhang \ref{Imp}) erstellte Textdatei zum Parameter $X_{it}$ trotz der zwei Indizes auslesen kann. Dies erfolgt durch einen \textit{If}-Befehl, wie der Ausschnitt (Quellcode \ref{rcpsp_a}) des Quellcodes \ref{rcpspc} aus dem Anhang \ref{rcodes} zeigt. Die Kapazitätsplanung ist mit dem Einlesen der Ergebnisse abgeschlossen.\\

\lstinputlisting[language=ruby, firstline=102, lastline=114, caption=Ausschnitt aus dem RoR-Controller für das RCPSP, style=Listing, label= rcpsp_a]{/Users/Superuser/as-rcpsp/app/controllers/rcpsps_controller.rb}

Über den Button \textit{Zurück zur Projektplanung} gelangt ein Admin zurück zur Verwaltungsseite. Sofern die Kostenplanung gewünscht ist, kann diese übe die Verwaltungsseite gestartet werden. Bevor der optimale Kostenplan für das vorhandene Projekt berechnet werden kann, muss zunächst äquivalent zur Einstellung des Starttermins eine Deadline eingerichtet werden. Dies funktioniert erneut über ein Kalendermenü des \glqq Bootstrap-Datepicker-Rails\grqq. Es sollte bei der Bestimmung der Deadline darauf geachtet werden, dass die Deadline in einem sinnvollen Verhältnis zum Starttermin steht. Eine zu kurze oder lange Zeitspanne zwischen den beiden Terminen kann zu unbrauchbaren Ergebnissen führen. Wurde eine geeignete Deadline ausgewählt und die Optimierung des Kostenplans gestartet, öffnet sich nach einer kurzen Rechenzeit die Übersicht der Ressourcen. Dieses erfolgt mit der Aktion \texttt{optimize2} und \texttt{solution2} des \texttt{rcpsps\_controller}.\\ 

Auf der Seite mit der Ressourcenübersicht sind die Projektkosten und die Zusatzkosten, die jede Ressource durch Einhaltung der Deadline verursachen, ausgelesen (siehe Abbildung \ref{ResAd}). Bei der Kostenplanung ist jedoch nicht nur relevant, wie hoch die Kosten zur Durchführung des Projektes sind, sondern auch die Zeitpunkte, zu denen die Vorgänge stattfinden. Um dies zu untersuchen, bietet sich dem Admin die Möglichkeit, ein weiteres Mal die Seite mit der Übersicht der Vorgänge aufzurufen. Auf dieser Seite ist angestoßen durch die Berechnung des optimalen Kostenplans der entsprechende kritische Pfad mit allen frühesten und spätesten Zeitpunkten in die Tabelle eingelesen. Die alten Ergebnisse der Kapazitätsplanung sind gelöscht und dadurch wird kein veralteter Wert angezeigt (siehe Abbildung \ref{VorKo}). Somit sind alle Informationen des Kostenplans einsehbar, die Kostenplanung ist ebenfalls abgeschlossen.\\  

\begin{figure}[h!]
  \begin{center}
    \includegraphics[width=120mm]{Bilder/Vorgaenge_Kostenpl.png}
    \caption{Ergebnis der Kapazitätsplanung}  \label{VorKo}
  \end{center}
\end{figure}

\subsection{Integration in die Web-Applikation und Beschreibung des Unterprogramms \textit{\glqq rgl\grqq}}\label{rgl}



\section{Kritische Würdigung des Anwendungssystems} \label{krit}

\section{Fazit} \label{Fazit}

\bibliographystyle{Prod_Seminar}    %legt die zu verwendende BIBTEX-Stildatei fest
\newpage
\bibliography{Literatur}    %an der Stelle zu verwenden, an der das Literaturverzeichnis gesetzt werden soll;
                            %Literatur ist der Dateiname der BIB-Datei mit den LiteraturLiteratur-Informationen

\newpage
%%%%%%%%%%%%%%%%%%%%%%%%%%%%%%%%%%%%%%%%%%%%%%%%%%%%%%
%
%    ggf. Anhang
%
%%%%%%%%%%%%%%%%%%%%%%%%%%%%%%%%%%%%%%%%%%%%%%%%%%%%%%
\begin{appendix}
\section{Anhang}

\subsection{Datenbankschema}\label{db-schema}

\begin{figure}[h!]
  \begin{center}
    \includegraphics[width=150mm]{Bilder/DB.pdf}
    \caption{Datenbankschema der Web-Applikation Projektplanung}  \label{schema}
  \end{center}
\end{figure}

\subsection{GAMS-Implementierung des Beispiels}\label{Imp}
%\lstinputlisting[language=ruby, firstline=37, lastline=45, caption=Name, style=Listing]{/Users/Superuser/as-rcpsp/RCPSP1.gms}
\lstinputlisting[language=ruby, caption=GAMS-Code zur Kapazitätsplanung, style=Listing, label=rcpsp1]{/Users/Superuser/as-rcpsp/RCPSP1.gms}

\lstinputlisting[language=ruby, caption=GAMS-Code zur Kostenplanung, style=Listing, label=rcpsp2]{/Users/Superuser/as-rcpsp/RCPSP2.gms}


%%%%%%%%%RoR Codes%%%%%%%%%%%%%%%%%%%%%%%%%%%%%
\subsection{Ruby on Rails Programmcodes}\label{rcodes}

%%%%%%GEMFILIE
\lstinputlisting[language=ruby, caption=Gemfile der Web-Applikation Projektplanung, style=Listing, label=gemfile]{/Users/Superuser/as-rcpsp/Gemfile}

%%%%%%%routes
\lstinputlisting[language=ruby, caption=Routes-Datei der Web-Applikation Projektplanung, style=Listing, label=routes]{/Users/Superuser/as-rcpsp/config/routes.rb}

%%%%%%%%%%%%%Controller
\lstinputlisting[language=ruby, caption=RoR-Controller für die Vorgangsrelationen, style=Listing, label=ppc]{/Users/Superuser/as-rcpsp/app/controllers/procedure_procedures_controller.rb}

\lstinputlisting[language=ruby, caption=RoR-Controller für die Vorgangs-Ressourcen-Kombinationen, style=Listing]{/Users/Superuser/as-rcpsp/app/controllers/procedure_resources_controller.rb}

\lstinputlisting[language=ruby, caption=RoR-Controller für die Vorgänge, style=Listing]{/Users/Superuser/as-rcpsp/app/controllers/procedures_controller.rb}

\lstinputlisting[language=ruby, caption=RoR-Controller für das Projekt, style=Listing]{/Users/Superuser/as-rcpsp/app/controllers/projects_controller.rb}

\lstinputlisting[language=ruby, caption=RoR-Controller für das RCPSP, style=Listing, label=rcpspc]{/Users/Superuser/as-rcpsp/app/controllers/rcpsps_controller.rb}

\lstinputlisting[language=ruby, caption=RoR-Controller für die Ressourcen, style=Listing]{/Users/Superuser/as-rcpsp/app/controllers/resources_controller.rb}

\lstinputlisting[language=ruby, caption=RoR-Controller für die statischen Seiten, style=Listing, label=spc]{/Users/Superuser/as-rcpsp/app/controllers/static_pages_controller.rb}

\lstinputlisting[language=ruby, caption=RoR-Controller für die Users, style=Listing]{/Users/Superuser/as-rcpsp/app/controllers/users_controller.rb}

%%%%MODELS
\lstinputlisting[language=ruby, caption=RoR-Modell für die Vorgangsrelationen, style=Listing]{/Users/Superuser/as-rcpsp/app/models/procedure_procedure.rb}

\lstinputlisting[language=ruby, caption=RoR-Modell für die Vorgangs-Ressourcen-Kombinationen, style=Listing, label=prm]{/Users/Superuser/as-rcpsp/app/models/procedure_resource.rb}

\lstinputlisting[language=ruby, caption=RoR-Modell für die Vorgänge, style=Listing]{/Users/Superuser/as-rcpsp/app/models/procedure.rb}

\lstinputlisting[language=ruby, caption=RoR-Modell für das Projekt, style=Listing]{/Users/Superuser/as-rcpsp/app/models/project.rb}

\lstinputlisting[language=ruby, caption=RoR-Modell für die Ressourcen, style=Listing]{/Users/Superuser/as-rcpsp/app/models/resource.rb}

\lstinputlisting[language=ruby, caption=RoR-Modell für die Users, style=Listing, label=fehler_code]{/Users/Superuser/as-rcpsp/app/models/user.rb}

%%%VIEWS propro
\lstinputlisting[language=ruby, caption=RoR-Seite für die Vorgangsrelationen - Formular, style=Listing]{/Users/Superuser/as-rcpsp/app/views/procedure_procedures/_form.html.erb}

\lstinputlisting[language=ruby, caption=RoR-Seite für die Vorgangsrelationen - Bearbeitung, style=Listing]{/Users/Superuser/as-rcpsp/app/views/procedure_procedures/edit.html.erb}

\lstinputlisting[language=ruby, caption=RoR-Seite für die Vorgangsrelationen - Übersicht, style=Listing]{/Users/Superuser/as-rcpsp/app/views/procedure_procedures/index.html.erb}

\lstinputlisting[language=ruby, caption=RoR-Seite für die Vorgangsrelationen - Erstellung, style=Listing]{/Users/Superuser/as-rcpsp/app/views/procedure_procedures/new.html.erb}

\lstinputlisting[language=ruby, caption=RoR-Seite für die Vorgangsrelationen - Anzeige, style=Listing]{/Users/Superuser/as-rcpsp/app/views/procedure_procedures/show.html.erb}

%%%VIEWS prores
\lstinputlisting[language=ruby, caption=RoR-Seite für die Vorgangs-Ressourcen-Kombinationen - Formular, style=Listing]{/Users/Superuser/as-rcpsp/app/views/procedure_resources/_form.html.erb}

\lstinputlisting[language=ruby, caption=RoR-Seite für die Vorgangs-Ressourcen-Kombinationen - Bearbeitung, style=Listing]{/Users/Superuser/as-rcpsp/app/views/procedure_resources/edit.html.erb}

\lstinputlisting[language=ruby, caption=RoR-Seite für die Vorgangs-Ressourcen-Kombinationen - Übersicht, style=Listing]{/Users/Superuser/as-rcpsp/app/views/procedure_resources/index.html.erb}

\lstinputlisting[language=ruby, caption=RoR-Seite für die Vorgangs-Ressourcen-Kombinationen - Erstellung, style=Listing]{/Users/Superuser/as-rcpsp/app/views/procedure_resources/new.html.erb}

\lstinputlisting[language=ruby, caption=RoR-Seite für die Vorgangs-Ressourcen-Kombinationen - Anzeige, style=Listing]{/Users/Superuser/as-rcpsp/app/views/procedure_resources/show.html.erb}

%%%VIEWS pro
\lstinputlisting[language=ruby, caption=RoR-Seite für die Vorgänge - Formular, style=Listing]{/Users/Superuser/as-rcpsp/app/views/procedures/_form.html.erb}

\lstinputlisting[language=ruby, caption=RoR-Seite für die Vorgänge - Bearbeitung, style=Listing]{/Users/Superuser/as-rcpsp/app/views/procedures/edit.html.erb}

\lstinputlisting[language=ruby, caption=RoR-Seite für die Vorgänge - Übersicht, style=Listing]{/Users/Superuser/as-rcpsp/app/views/procedures/index.html.erb}

\lstinputlisting[language=ruby, caption=RoR-Seite für die Vorgänge - Erstellung, style=Listing]{/Users/Superuser/as-rcpsp/app/views/procedures/new.html.erb}

\lstinputlisting[language=ruby, caption=RoR-Seite für die Vorgänge - Anzeige, style=Listing]{/Users/Superuser/as-rcpsp/app/views/procedures/show.html.erb}

%%%VIEWS res
\lstinputlisting[language=ruby, caption=RoR-Seite für die Ressourcen - Formular, style=Listing]{/Users/Superuser/as-rcpsp/app/views/resources/_form.html.erb}

\lstinputlisting[language=ruby, caption=RoR-Seite für die Ressourcen - Tabelle als unangemeldeter User, style=Listing]{/Users/Superuser/as-rcpsp/app/views/resources/_free.html.erb}

\lstinputlisting[language=ruby, caption=RoR-Seite für die Ressourcen - Tabelle als angemeldeter User, style=Listing, label=sign_res]{/Users/Superuser/as-rcpsp/app/views/resources/_signed_in.html.erb}

\lstinputlisting[language=ruby, caption=RoR-Seite für die Ressourcen - Bearbeitung, style=Listing]{/Users/Superuser/as-rcpsp/app/views/resources/edit.html.erb}

\lstinputlisting[language=ruby, caption=RoR-Seite für die Ressourcen - Übersicht, style=Listing, label=index_res]{/Users/Superuser/as-rcpsp/app/views/resources/index.html.erb}

\lstinputlisting[language=ruby, caption=RoR-Seite für die Ressourcen - Erstellung, style=Listing]{/Users/Superuser/as-rcpsp/app/views/resources/new.html.erb}

\lstinputlisting[language=ruby, caption=RoR-Seite für die Ressourcen - Anzeige, style=Listing]{/Users/Superuser/as-rcpsp/app/views/resources/show.html.erb}

%%%VIEWS static
\lstinputlisting[language=ruby, caption=RoR-Seite für die Optimierungsseite zur Projektplanung, style=Listing]{/Users/Superuser/as-rcpsp/app/views/static_pages/rcpsp.html.erb}

\lstinputlisting[language=ruby, caption=RoR-Seite für die Startseite, style=Listing, label=test]{/Users/Superuser/as-rcpsp/app/views/static_pages/home.html.erb}

\lstinputlisting[language=ruby, caption=Kopfzeile der Web-Apllikation, style=Listing, label=header]{/Users/Superuser/as-rcpsp/app/views/layouts/_header.html.erb}


%%%VIEWS user
\lstinputlisting[language=ruby, caption=RoR-Seite bzgl. der Lösung von Usern über die Übersichtsseite, style=Listing]{/Users/Superuser/as-rcpsp/app/views/users/_user.html.erb}

\lstinputlisting[language=ruby, caption=RoR-Seite für die User - Bearbeitung, style=Listing]{/Users/Superuser/as-rcpsp/app/views/users/edit.html.erb}

\lstinputlisting[language=ruby, caption=RoR-Seite für die User - User-/Mitarbeiterübersicht, style=Listing]{/Users/Superuser/as-rcpsp/app/views/users/index.html.erb}

\lstinputlisting[language=ruby, caption=RoR-Seite für die User - Erstellung, style=Listing]{/Users/Superuser/as-rcpsp/app/views/users/new.html.erb}

\lstinputlisting[language=ruby, caption=RoR-Seite für die User - Anzeige, style=Listing, label=show_user]{/Users/Superuser/as-rcpsp/app/views/users/show.html.erb}

%%%schema
\lstinputlisting[language=ruby, caption=RoR-Datenbankschema, style=Listing]{/Users/Superuser/as-rcpsp/db/schema.rb}

%%%sample
\lstinputlisting[language=ruby, caption=Beispieldaten für die Datenbank, style=Listing]{/Users/Superuser/as-rcpsp/lib/tasks/sample_data.rake}

%%%datepicker
%\lstinputlisting[language=java, caption=JavaScrip-Datei zum Datepicker, style=Listing2]{/Users/Superuser/as-rcpsp/vendor/assets/javascripts/bootstrap-datepicker/core.js}

\end{appendix}
\end{document}
%%%%%%%%%%%%%%%%%%%%%%%%%%ENDE%%%%%%%%%%%%%%%%%%%

