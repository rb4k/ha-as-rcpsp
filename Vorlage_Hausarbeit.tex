\documentclass[a4paper,12pt,parskip,bibtotoc,liststotoc]{article}
    %Festlegung der Dokumentenklasse, zahlreiche Vereinbarungen über Layout, Gliederungsstrukturen,
    %bsp. article -> section, subsection..., book -> chapter, section...
    %parskip = Abstand zwischen Absätzen, Veränderung durch \setlength
\usepackage[utf8]{inputenc}   %Eingabezeichencodierung, die direkte Tastatureingabe von Umlauten ist möglich
\usepackage[ngerman]{babel}     %Neu deutsche Rechtschreibung, Umlaute können geschrieben werden
\usepackage{setspace}           %für Zeilenabstand
\usepackage[notindex,nottoc]{tocbibind}   %Inhaltsverzeichnisse erstellen

%zusätzliche benötigte Pakete
\usepackage{graphicx}           %Graphik
\usepackage{amsmath,amssymb}    %Mathematik
\usepackage{natbib}             %Zitate
\usepackage{marvosym}           %enthält Symbole wie das Eurozeichen
\usepackage{eurosym}
%\setcounter{secnumdepth}{3}
%\setcounter{tocdepth}{3}
\usepackage{footmisc}

\usepackage{amsmath}
%\usepackage{mathtools} % lädt auch amsmath
\usepackage{moreverb}

\usepackage{amsthm}
\newtheorem{mydef}{Definition}

\usepackage[table]{xcolor}
\usepackage{colortbl}
\usepackage{multirow}
\usepackage{mdwlist}   %Verringerung Abstand zwischen items -> \begin{itemize*} \end{itemize*}
\usepackage[labelsep=space,justification=centering]{caption}

%\usepackage{hyperref}  %erlaubt Links innerhalb des pdf-Dokuments zu erzeugen

\setlength{\parindent}{0pt}     %Verhinderung des horizontalen Einrückens zu Beginn eines Absatzes

%Seitenlayout
\topmargin -0.9cm       %Vertikaler Abstand der Kopfzeile von der Bezugslinie
\textheight 25cm        %Abstand der Grundlinie der Kopfzeile zum Haupttext
\textwidth 16.5cm       %Breite des Haupttexts
\footskip 1cm           %Abstand der Grundlinien der letzten Textzeile und der Fußzeile
\voffset -0.5cm         %Vertikale Bezugspunktposition
\hoffset -1.2cm         %Horizontale Bezugspunktposition

\onehalfspacing         %anderthalbzeiliger Abstand

\newcommand{\url}{\;}   %URL im Literaturverzeichnis

%eigene Befehlsdefinitionen
\newcommand{\be}{\begin{equation}}     %Mathematische Umgebung
\newcommand{\ee}{\end{equation}}
\newcommand{\bea}{\begin{eqnarray}}
\newcommand{\eea}{\end{eqnarray}}
\newcommand{\bean}{\begin{eqnarray*}}  %ohne Nummerierung
\newcommand{\eean}{\end{eqnarray*}}    %ohne Nummerierung
%%%%%%%%%%%%%%%%%%%%%%%%%%%%%%%%%%%%%%%%%%%%%%%%%%


%%%%%%%%%%%%%%%%%%%%%%%%%%%%%%%%%%%%%%%%%%%%%%%%%%%%%%
%
%    Anfang des Textes
%
%%%%%%%%%%%%%%%%%%%%%%%%%%%%%%%%%%%%%%%%%%%%%%%%%%%%%%
\begin{document}

\pagenumbering{roman}  %römische Seitennummerierung
%%%%%%%%%%%%%%%%%%%%%%%%%%%%%%%%%%%%%%%%%%%%%%%%%%%%%%
%
%    Titelseite
%
%%%%%%%%%%%%%%%%%%%%%%%%%%%%%%%%%%%%%%%%%%%%%%%%%%%%%%
\thispagestyle{empty}  %keine Seitenzahl auf Titelseite
Leibniz Universität Hannover\\
Wirtschaftswissenschaftliche Fakultät\\
Institut für Produktionswirtschaft\\
Prof.\ Dr.\ Stefan Helber

\vspace{5cm}

\begin{center}
Hausarbeit im Rahmen der Veranstaltung \\
Entwicklung von Anwendungssystemen  im WiSe 2014/2015 \\
(Veranstaltungs-Nr. 173610)

\vspace{2.5cm}

%Thema Nr. 5\\[1mm]    %hier Themennummer eintragen
{\Large RCPSP \\
RCPSP}
\end{center}

\vspace{5.5cm}


\begin{table}[h!]
    \vspace*{-3mm}
    \hspace*{2mm}
  \renewcommand{\arraystretch}{1,5}
    \begin{tabular}{ll}
Andreas Hipp &Robert Matern \\
Adresse&Plathnerstr. 49 \\
PLZ Ort&30175 Hannover \\
Matr.-Nr. ???&Matr.-Nr. 2798160 \\[3mm]
Abgabedatum: 24.03.2015
	\end{tabular}
\end{table}

\newpage

%Inhaltsverzeichnis erstellen
\tableofcontents

\newpage  %neue Seite

%Abbildungsverzeichnis erstellen
\listoffigures

%Tabellenverzeichnis erstellen
\listoftables
\newpage
%Abkürzungsverzeichnis
\section*{Abkürzungsverzeichnis}
\addcontentsline{toc}{section}{Abkürzungsverzeichnis}
\begin{table}[h!]
    \vspace*{-3mm}
    \hspace*{2mm}
  \renewcommand{\arraystretch}{1,5}
    \begin{tabular}{ll}  %hier die Spaltenausrichtung und Anzahl eintragen
           RCPSP      & Resource-Constrained Project Scheduling \\
SGS & Schedule Generation Scheme\\
	\end{tabular}
\end{table}
\newpage
%Symbolverzeichnis
\section*{Symbolverzeichnis}
\addcontentsline{toc}{section}{Symbolverzeichnis}
\begin{table}[h!]
    \vspace*{-3mm}
        \hspace*{2mm}
      \renewcommand{\arraystretch}{1,5}
    \begin{tabular}{ll} 
$d_i$ & Dauer von Vorgang $i$ \\
$FE_i$& frühestes Ende von Vorgang $i$\\
$i,h=1,...,I$ & Vorgänge \\
$k_{ir}$& Kapazitätsbedarf von Vorgang $i$ auf Ressource $r$\\
$kp_r$ & verfügbare Kapazität von Ressource $r$ je Periode\\
$\mathcal{N}_i$ & Menge der direkten Nachfolger von Vorgang $i$ \\
$r=1,...,R$ & Ressourcen \\
$SE_i$& spätestes Ende von Vorgang $i$\\
$t,\tau=1,..., T$ & Perioden\\
$\mathcal{V}_i$ & Menge der direkten Vorgänger von Vorgang $i$ \\
$X_{jt}\in\{0,1\}$ & gleich $1$, falls Vorgang $j$ in Periode $t$ endet, sonst $0$
  	\end{tabular}
\end{table}
\newpage
\pagenumbering{arabic}   %ab hier arabische Seitenzahlen beginnend mit 1

%%%%%%%%%%%%%%%%%%%%%%Textteil%%%%%%%%%%%%%%%%%%%%%%%%%%

\section{Einleitung} \label{start}
Bereits seit mehreren Dekaden spielt Projektarbeit eine wichtige Rolle bei der Aufgabenabwicklung in Wirtschaft und Verwaltung.\footnote{Vgl. \cite{zimmermann2006projektplanung}, S. VI} Dabei wird unter dem Begriff Projekt verstanden:

\begin{mydef}
\glqq Ein Projekt ist eine zeitlich befristete, relativ innovative und risikobehaftete Aufgabe von erheblicher Komplexität, die aufgrund ihrer Schwierigkeit und Bedeutung meist ein gesondertes Projektmanagement erfordert.\grqq\footnote{Vgl. \cite{projektdef}}
\end{mydef}

Laut dieser Definition geht eine zeitliche Restriktion mit einem Projekt einher. Durch das Zerlegen des Projekts in einzelne Arbeitsgänge wird versucht die Komplexität zu reduzieren und eine geordnete Abfolge der Arbeitsgänge zu erstellen, um das Projektziel zu erreichen.\footnote{Vgl. \cite{zimmermann2006projektplanung}, S. 4} Projektziele können unterschiedlich kategorisiert werden, z. B. in Sach-, Termin- oder Kostenziele.\footnote{Vgl. \cite{felkai2011analysieren}, S. 52}

Nach DIN 69900 hat ein Arbeitsgang eines Projekts einen definierten Anfang sowie ein definiertes Ende und dient für das Projekt als Ablaufelement zur Beschreibung eines bestimmten Geschehens.\footnote{Vgl. \cite{69900D}, S. 15} Trotz der Zerlegung besitzen die einzelnen Arbeitsgänge des Projekts eine Beziehung, mit der die Reihenfolge der Ablauffolge bestimmbar ist.\footnote{Vgl. \cite{kellenbrink2014einfuhrung}, S. 6-7} %D. h. es können nur Arbeitsgänge abgeschlossen werden, wenn deren notwendigen Vorgänge bereits abgeschlossen wurden. Ein einfaches Beispiel wäre die sogenannte Hochzeit in der Automobilherstellung. Sobald Karosserie und Motor eines Fahrzeugs hergestellt wurden, können diese zwei Elemente in einem nachfolgenden Prozess verbunden werden.
Ein Arbeitsgang ist i. d. R. verbunden mit dem Einsatz von Ressourcen, welche wiederum mit Kosten verbunden sind. Dementsprechend versucht ein effizientes Projektmanagement, neben der Einhaltung des Projektziels, auch den Einsatz der Ressourcen zu minimieren.\footnote{Vgl. \cite{bartels2009projektplanung}, S. 11-12}

\begin{mydef}
\glqq Allgemein bezeichnet der Begriff Projektmanagement alle leitenden und administrativen Aktivitäten, die zur Durchführung eines Projektes notwendig sind. Es beschreibt die Gesamtheit von Führungsaufgaben, -organi\-sation, -techniken und -mitteln zur zielorientierten Durchführung großer Vorhaben.\grqq\footnote{Vgl. \cite{hering2014projektmanagement}, S. 1-3; in Anlehnung an DIN 69901 und ISO 21500:2012-09}
\end{mydef}

Eine Möglichkeit, das Projektziel unter minimaler Ressourcenverwendung zu erreichen, ist die effiziente Planung der Ablauffolge der Arbeitsgänge eines Projekts.\footnote{Vgl. \cite{bartels2009projektplanung}, S. 11-12} Damit ist es möglich, mehrere Projekte bei einer gegebenen Zeitvorgabe unter Einhaltung von Ressourcenrestriktionen fertigzustellen bzw. bei konstanter Ressourcenkapazität ein Projekt in kürzerer Zeit abzuschließen. Dementsprechend von großer Bedeutung ist die Projektplanung im Projektmanagement.\footnote{Vgl. \cite{zimmermann2006projektplanung}, S. VI\label{zum}}%Beispielweise kann ein Automobilhersteller durch Optimierung des Produktentstehungsprozess durch effizientes planen seiner vorhandenen Ressourcen in einer vordefinierten Zeit eine größere Anzahl an Fahrzeugen entwickeln, als wenn das Unternehmen keine effiziente Projektplanung verfolgt. 

\begin{mydef}
\glqq Projektplanung ist die Planung aller Arbeitsgänge eines Projekts durch Zuweisung eines Startzeitpunktes, so dass die Zeitbeziehung zwischen den Vorgängen eingehalten und knappe Ressourcenkapazitäten nicht überschritten werden.\grqq\footref{zum}
\end{mydef}

Eine effiziente Projektplanung reduziert die gesamte Fertigstellungsdauer eines Projekts. Es wird eine wirkungsvolle Gestaltung des Verbrauchs der zur Verfügung stehenden Ressourcen über die Laufzeit des Projekts ermöglicht. Somit handelt es sich um ein mathematisches Optimierungsproblem, bei dem für ein Projekt die Ressourcenbeschränkung über die Laufzeit einzuhalten ist und die Fertigstellungszeit minimiert werden soll.
%Für das ressourcenbeschränkte Projektplanungsproblem gibt es Verfahren der exakten und heuristischen Lösung. In der Praxis, wie auch in dieser Arbeit \footnote{Vgl. Kapitel \ref{notwendig}}, wird i. d. R. auf Heuristiken zurückgegriffen.\footnote{Vgl. \cite{herroelen2005project}, S. 420}
%Eine typische Ressource in der Projektplanung ist der Faktor Zeit, da lt. Definition ein Projekt zeitlich befristet ist. % Andere Arten von Ressourcen sind dürfen bei der Planung von Projekten jedoch nicht vernachlässigt werden.
Ziel der vorliegenden Arbeit ist es...

\section{Mathematische Modellformulierung des ressourcenbeschränkten Projektplanungsproblems} \label{Grund}
Zur Sicherstellung des Planungserfolgs mittels Terminierung eines Projekts muss neben der Reihenfolgerestriktion auch der Ressourcenbedarf der unterschiedlichen Arbeitsgänge sichergestellt werden, da Projekte meist ein beschränktes Ressourcenkontingent haben.\footnote{Vgl. \cite{kellenbrink2014einfuhrung}, S. 11} Mit der Einhaltung des Ressourcenbedarfs ist es für das Projekt möglich, alle in Bearbeitung befindlichen Arbeitsgänge auszuführen und folgerichtig das Projekt abzuschließen. Neben limitierten Ressourcen, die während des gesamten Projekts nur ein Mal zur Verfügung stehen, wie bspw. das Projektbudget, gibt es Ressourcen, die nach einer bestimmten Anzahl von Perioden erneuert werden können.\footnote{Vgl. \cite{neumann2003project}, S. 21-22} Erneuerbare Ressourcen sind bspw. die Produktionskapazität einer Maschine oder der Personaleinsatz für ein Projekt. In dieser Arbeit wird der Fokus auf erneuerbare Ressourcen gelegt.\\

Zur Lösung des ressourcenbeschränkten Projektplanungsproblems kann das Modell \textit{Resource-Constrained Project Scheduling Problem (RCPSP)} genutzt werden. Das Modell RCPSP legt durch Festlegung der Aktivitätsstartzeitpunkte unter Einhaltung der Startzeitpunkt- bzw. der Vorrangsbedingung der einzelnen Arbeitsgänge sowie der Kapazitätsbeschränkung der erneuerbaren Ressourcen den Projektgrundablauf zur Zielerreichung der Minimierung der Projektdauer fest.\footnote{Vgl. \cite{demeulemeester2011robust}, S. 23} Die Zielfunktion des RCPSP zur Minimierung der Projektdauer ist weit verbreitet,\footnote{Vgl. \cite{drexl1997neuere}, S. 98} andere Varianten sind aber ebenfalls möglich.\footnote{Vgl. \cite{talbot1982resource}, S. 1200}\\

Nachfolgend wird das deterministische RCPSP in diskreter Zeit formuliert.\footnote{????} Da es sich um eine mathematische Modellformulierung in diskreter Zeit handelt, sind die Zeiteinheiten gleich den Perioden $t, \tau$.\\
\textbf{Modell RCPSP}
\begin{eqnarray} \label{Ziel}
\min Z = \sum_{t=FE_{I}}^{SE_{I}}t \cdot X_{I,t}\hfill  
\end{eqnarray}

unter Beachtung der Restriktionen
\begin{multline} \label{N1}
\sum_{t=FE_{i}}^{SE_{i}} X_{it} = 1
\hfill   i = 1,...,I
\end{multline}\vspace{-3.0ex}

\begin{multline} \label{N2}
\sum_{t=FE_{h}}^{SE_{h}}t \cdot X_{ht} \leq \sum_{t=FE_{i}}^{SE_{i}}(t - d_{i}) \cdot X_{it}
\hfill   i =1,...,I;\; h \in \mathcal{V}_{i}
\end{multline}\vspace{-3.0ex}

\begin{multline} \label{N3}
\sum_{i=1}^{I}\sum_{\tau=\max(t,FE_{i})}^{\tau=\min(t+d_i-1,SE_i)}k_ {ir} \cdot X_{i\tau} \leq kp_{r}
\hfill   r =1,...,R;\; t=1,...,T
\end{multline}\vspace{-3.0ex}
\begin{multline} \label{N4}
X_{it} \in \{0,1\}
\hfill   i \in \mathcal{I};\; t \in \{FE_{i},...,SE_{i}\}\end{multline}\vspace{-6.0ex}\\

Ein Projekt hat $I$ unterschiedliche Arbeitsgänge. Jeder Arbeitsgang $i$ hat eine definierte Menge von zu erledigenden Vorgängerarbeitsgängen $h \in \mathcal{V}_{i}$ und die Arbeitsgänge müssen zur Fertigstellung des Projekts topologisch abgearbeitet werden. D. h. der Vorgänger $h$ hat stets eine kleinere Ordnungszahl als sein Nachfolger $i\;(h<i)$ und muss für den weiteren Projektverlauf beendet sein. Die Bearbeitungsdauer eines Arbeitsgangs $i$ wird mit dem Parameter $d_{i}$ festgelegt.  Bei dem RCPSP in diskreter Zeit wird die Annahme getroffen, dass die Dauer ganzzahlig ist. Der Startzeitpunkt des Projekts ist $t = 0$ und erstreckt sich über einen Gesamtzeitraum von $T$ Perioden. Um die Reihenfolgebedingungen einzuhalten, werden einem Projekt die zwei Dummy-Arbeitsgänge \glqq Beginn\grqq\;($i=1$) und \glqq Ende\grqq\;($i=I$) hinzugefügt, welche mit einer Dauer von $0$ Zeiteinheiten bewertet werden.\footnote{Vgl. \cite{zimmermann2006projektplanung}, S. 66} Die benötigten Kapazitäten der erneuerbaren Ressource $r$ bei Durchführung von Arbeitsgang $i$ wird durch $k_{ir}$ definiert. Die Ressourcen $r \in R$ sind in einer Periode innerhalb des Umfangs ihrer Kapazität $kp_{r}$ nutzbar. Da es sich um erneuerbare Ressourcen handelt, sind diese zu jeder neuen Periode in vollem Umfang erneut nutzbar, wobei nichtverbrauchte Ressourcen nicht auf nachfolgende Perioden übertragbar sind.\footnote{Vgl. \cite{kellenbrink2014einfuhrung}, S. 12} Der Modellformulierung in diskreter Zeit wird die binäre Entscheidungsvariable $X_{it}$ hinzugefügt, damit der Fertigstellungszeitpunkt der einzelnen Arbeitsgänge $i$ festgelegt werden kann.\footnote{Vgl. \cite{pritsker1969multiproject}, S. 94} \\%Die Binärvariable $x_{jt}$ wird für den Zeitraum vom frühesten Fertigstellungszeitpunkt $EFT_{j}$ bis zum spätesten Fertigstellungszeitpunkt $LFT_{j}$ vom jeweiligen Arbeitsgang $j$ definiert.

Mit der Zielfunktion \eqref{Ziel} wird der Fertigstellungszeitpunkt des Projekts minimiert. Die Nebenbedingung \eqref{N1} sorgt dafür, dass ein Arbeitsgang $i$ nur jeweils zwischen dem frühesten und dem spätesten Fertigstellungszeitpunkt einmalig durchgeführt wird. Die Reihenfolgerestriktion wird mit der Nebenbedingung \eqref{N2} eingehalten. Sie stellt sicher, dass jeder Vorgänger $h \in \mathcal{V}_{i}$ beendet ist, bevor der Arbeitsgang $i$ startet.
Der Parameter $kp_{r}$ legt die Kapazitätsgrenze für eine erneuerbare Ressource $r$ je Periode $t$ fest. Mit der Nebenbedingung \eqref{N3} wird dieses zum einen formal dargestellt und zum anderen wird der Ressourcenverzehr während der gesamten Bearbeitungsdauer der Fertigstellung beachtet.
Mit der Nebenbedingung \eqref{N4} wird die Binärvariable $X_{it}$ für den Zeitraum $t = \{FE_{i},...,SE_{i}\}$ definiert, da aufgrund der Reihenfolgebeziehung \eqref{N2} der jeweils betrachtete Arbeitsgang nur in diesem Zeitraum fertiggestellt werden kann. Die gemischt-ganzzahlige Modellformulierung lässt sich durch Standard-Lösungsverfahren exakt lösen.\footnote{z. B. mittels eines Branch-and-Bound-Verfahrens, Vgl. \cite{kellenbrink2014einfuhrung}, S. 14}


\section{Lösung des RCPSP mittels Ruby on Rail und LP-Relaxation?!?} \label{Grund}

\section{Schlussbemerkungen}

\bibliographystyle{Prod_Seminar}    %legt die zu verwendende BIBTEX-Stildatei fest
\newpage
\bibliography{Literatur}    %an der Stelle zu verwenden, an der das Literaturverzeichnis gesetzt werden soll;
                            %Literatur ist der Dateiname der BIB-Datei mit den LiteraturLiteratur-Informationen

\newpage
%%%%%%%%%%%%%%%%%%%%%%%%%%%%%%%%%%%%%%%%%%%%%%%%%%%%%%
%
%    ggf. Anhang
%
%%%%%%%%%%%%%%%%%%%%%%%%%%%%%%%%%%%%%%%%%%%%%%%%%%%%%%
\begin{appendix}
\section{Anhang}

\subsection{GAMS-Implementierung des Beispiels}\label{Imp}

\subsection{Ruby on Rails Programmcode}\label{Anhang2}


\end{appendix}
\end{document}
%%%%%%%%%%%%%%%%%%%%%%%%%%ENDE%%%%%%%%%%%%%%%%%%%

