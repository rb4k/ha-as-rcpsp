\documentclass[a4paper,12pt,parskip,bibtotoc,liststotoc]{article}
    %Festlegung der Dokumentenklasse, zahlreiche Vereinbarungen �ber Layout, Gliederungsstrukturen,
    %bsp. article -> section, subsection..., book -> chapter, section...
    %parskip = Abstand zwischen Abs�tzen, Ver�nderung durch \setlength
\usepackage[latin1]{inputenc}   %Eingabezeichencodierung, die direkte Tastatureingabe von Umlauten ist m�glich
\usepackage[ngerman]{babel}     %Neu deutsche Rechtschreibung, Umlaute k�nnen geschrieben werden
\usepackage{setspace}           %f�r Zeilenabstand
\usepackage[notindex,nottoc]{tocbibind}   %Inhaltsverzeichnisse erstellen

%zus�tzliche ben�tigte Pakete
\usepackage{graphicx}           %Graphik
\usepackage{amsmath,amssymb}    %Mathematik
\usepackage{natbib}             %Zitate
\usepackage{marvosym}           %enth�lt Symbole wie das Eurozeichen
\usepackage{eurosym}
%\setcounter{secnumdepth}{3}
%\setcounter{tocdepth}{3}
\usepackage{footmisc}

\usepackage{amsmath}
%\usepackage{mathtools} % l�dt auch amsmath
\usepackage{moreverb}

\usepackage[table]{xcolor}
\usepackage{colortbl}
\usepackage{multirow}
\usepackage{mdwlist}   %Verringerung Abstand zwischen items -> \begin{itemize*} \end{itemize*}
\usepackage[labelsep=space,justification=centering]{caption}

%\usepackage{hyperref}  %erlaubt Links innerhalb des pdf-Dokuments zu erzeugen

\setlength{\parindent}{0pt}     %Verhinderung des horizontalen Einr�ckens zu Beginn eines Absatzes

%Seitenlayout
\topmargin -0.9cm       %Vertikaler Abstand der Kopfzeile von der Bezugslinie
\textheight 25cm        %Abstand der Grundlinie der Kopfzeile zum Haupttext
\textwidth 16.5cm       %Breite des Haupttexts
\footskip 1cm           %Abstand der Grundlinien der letzten Textzeile und der Fu�zeile
\voffset -0.5cm         %Vertikale Bezugspunktposition
\hoffset -1.2cm         %Horizontale Bezugspunktposition

\onehalfspacing         %anderthalbzeiliger Abstand

\newcommand{\url}{\;}   %URL im Literaturverzeichnis

%eigene Befehlsdefinitionen
\newcommand{\be}{\begin{equation}}     %Mathematische Umgebung
\newcommand{\ee}{\end{equation}}
\newcommand{\bea}{\begin{eqnarray}}
\newcommand{\eea}{\end{eqnarray}}
\newcommand{\bean}{\begin{eqnarray*}}  %ohne Nummerierung
\newcommand{\eean}{\end{eqnarray*}}    %ohne Nummerierung
%%%%%%%%%%%%%%%%%%%%%%%%%%%%%%%%%%%%%%%%%%%%%%%%%%


%%%%%%%%%%%%%%%%%%%%%%%%%%%%%%%%%%%%%%%%%%%%%%%%%%%%%%
%
%    Anfang des Textes
%
%%%%%%%%%%%%%%%%%%%%%%%%%%%%%%%%%%%%%%%%%%%%%%%%%%%%%%
\begin{document}

\pagenumbering{roman}  %r�mische Seitennummerierung
%%%%%%%%%%%%%%%%%%%%%%%%%%%%%%%%%%%%%%%%%%%%%%%%%%%%%%
%
%    Titelseite
%
%%%%%%%%%%%%%%%%%%%%%%%%%%%%%%%%%%%%%%%%%%%%%%%%%%%%%%
\thispagestyle{empty}  %keine Seitenzahl auf Titelseite
Leibniz Universit�t Hannover\\
Wirtschaftswissenschaftliche Fakult�t\\
Institut f�r Produktionswirtschaft\\
Prof.\ Dr.\ Stefan Helber

\vspace{5cm}

\begin{center}
Hausarbeit im Rahmen der Veranstaltung \\
Entwicklung von Anwendungssystemen  im WiSe 2014/2015 \\
(Veranstaltungs-Nr. 173610)

\vspace{2.5cm}

%Thema Nr. 5\\[1mm]    %hier Themennummer eintragen
{\Large RCPSP \\
RCPSP}
\end{center}

\vspace{5.5cm}


\begin{table}[h!]
    \vspace*{-3mm}
    \hspace*{2mm}
  \renewcommand{\arraystretch}{1,5}
    \begin{tabular}{ll}
Andreas Hipp &Robert Matern \\
Adresse&Plathnerstr. 49 \\
PLZ Ort&30175 Hannover \\
Matr.-Nr. ???&Matr.-Nr. 2798160 \\[3mm]
Abgabedatum: 24.03.2015
	\end{tabular}
\end{table}

\newpage

%Inhaltsverzeichnis erstellen
\tableofcontents

\newpage  %neue Seite

%Abbildungsverzeichnis erstellen
\listoffigures

%Tabellenverzeichnis erstellen
\listoftables
\newpage
%Abk�rzungsverzeichnis
\section*{Abk�rzungsverzeichnis}
\addcontentsline{toc}{section}{Abk�rzungsverzeichnis}
\begin{table}[h!]
    \vspace*{-3mm}
    \hspace*{2mm}
  \renewcommand{\arraystretch}{1,5}
    \begin{tabular}{ll}  %hier die Spaltenausrichtung und Anzahl eintragen
        %DLP			& Deterministisch-lineares Programm\\
	\end{tabular}
\end{table}
\newpage
%Symbolverzeichnis
\section*{Symbolverzeichnis}
\addcontentsline{toc}{section}{Symbolverzeichnis}
\begin{table}[h!]
    \vspace*{-3mm}
        \hspace*{2mm}
      \renewcommand{\arraystretch}{1,5}
    \begin{tabular}{ll} 
    %$\textbf{a}_{m}$ & Vektor des Ressourcenverbrauchs im Ausf�hrungsmodus $m$\\


	\end{tabular}
\end{table}
\newpage
\pagenumbering{arabic}   %ab hier arabische Seitenzahlen beginnend mit 1

%%%%%%%%%%%%%%%%%%%%%%Textteil%%%%%%%%%%%%%%%%%%%%%%%%%%

\section{Einleitung} \label{start}


\section{Das ressourcenbeschr�nkte Projektplanungsproblem} \label{Grund}

\section{L�sung des RCPSP mittels Ruby on Rail und LP-Relaxation?!?} \label{Grund}

\section{Schlussbemerkungen}

\bibliographystyle{Prod_Seminar}    %legt die zu verwendende BIBTEX-Stildatei fest
\newpage
\bibliography{Literatur}    %an der Stelle zu verwenden, an der das Literaturverzeichnis gesetzt werden soll;
                            %Literatur ist der Dateiname der BIB-Datei mit den LiteraturLiteratur-Informationen

\newpage
%%%%%%%%%%%%%%%%%%%%%%%%%%%%%%%%%%%%%%%%%%%%%%%%%%%%%%
%
%    ggf. Anhang
%
%%%%%%%%%%%%%%%%%%%%%%%%%%%%%%%%%%%%%%%%%%%%%%%%%%%%%%
\begin{appendix}
\section{Anhang}

\subsection{GAMS-Implementierung des Beispiels}\label{Imp}

\subsection{Ruby on Rails Programmcode}\label{Anhang2}


\end{appendix}
\end{document}
%%%%%%%%%%%%%%%%%%%%%%%%%%ENDE%%%%%%%%%%%%%%%%%%%

